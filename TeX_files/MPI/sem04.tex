\section{Semaine 04 (07/10-11/10) }


\e{Notions abordées :}
\begin{itemize}
	\item Électrocinétique.
	\item Mécanique de MPSI.
\end{itemize}

\subsection{Exercice 1 : Système à deux ressorts}

On considère une masse $m$ astreinte à se déplacer sur un axe horizontal $(Ox)$ et fixée à une paroi à gauche et une à droite par deux ressorts identiques $(k, l_0)$. Les parois sont distantes de $L$.

\begin{enumerate}
	\item Appliquer le principe fondamental de la dynamique à la masse $m$.
	\item En déduire la position d'équilibre $x_e$.
	\item Étudier les petites oscillations autour de la position d'équilibre.
	\item On envisage l'existence d'un frottement fluide d'intensité proportionnelle à la vitesse via une constante $\beta$. Établir l'équation différentielle du mouvement. Pour quelles valeurs de $\beta$ la masse $m$ oscille-t-elle ?
	\item Comment choisir $\beta$ pour un retour le plus rapide à la position d'équilibre. Quel est le temps caractéristique d'amortissement ?
\end{enumerate}

\subsection{Exercice 2 : Frottement et facteur de qualité}

On considère un ressort horizontal de constante de raideur $k$ et de longueur à vide $l_0$. Une extrémité du ressort est fixe et l'autre attachée à un mobile de masse $m$. Le mobile subit une force de frottement fluide proportionnelle à sa vitesse via une constante $\beta$.

\begin{enumerate}
	\item Déterminer l'équation différentielle du mouvement. Introduire une pulsation propre et un facteur de qualité.
	\item Résoudre l'équation différentielle. Simplifier l'expression dans le cas $Q \gg 1$.
	\item En déduire que $Q$ est une bonne approximation du nombre d'oscillations avant le retour à l'équilibre.
	\item On considère maintenant l'énergie mécanique relative perdue sur une pseudo-période. L'exprimer en fonction de $Q$. 
	\item On considère maintenant un opérateur qui impose une force $\vec{F(t)} = m A \cos{\omega t} \vec{e_x}$. Déterminer la fonction de transfert du système et interpréter $Q$ d'une nouvelle façon.
\end{enumerate}

\subsection{Exercice 3 : Mouvement autour d'une position d'équilibre}

Soit un point matériel de masse $m$ astreint à se déplacer selon un axe $(Ox)$ et d'énergie potentielle $E_p(x) = \frac{-a}{x^2} + \frac{b}{x^3}$ avec $a, b > 0$.

\begin{enumerate}
	\item Montrer en général qu'une position d'équilibre correspond à un extremum local d'énergie potentiel. À quelle condition une position d'équilibre est-elle stable ? instable ?
	\item Tracer le profil d'énergie potentiel.
	\item Déterminer la ou les position(s) d'équilibre ainsi que leur stabilité.
	\item Étudier les petites oscillations autour de la position d'équilibre stable.
	\item Déterminer, dans le cas d'une énergie potentielle générale, l'expression de la pulsation des petites oscillations.
\end{enumerate}
