\section{Semaine 08 (18/11-22/11) }


\e{Notions abordées :}
\begin{itemize}
	\item Transformations chimiques d'un système.
	\item Acides et bases, réactions acide-base.
	\item Réaction d'oxydoréduction et piles.
	\item Dosages.
\end{itemize}

\subsection{Questions de cours}

\begin{enumerate}
	\item Réaliser le schéma d'une pile. 
	\item Écrire une réaction d'oxydoréduction entre l'ion cuivre au degré d'oxydation $2$ et l'argent solide.
	\item Écrire l'équation d'une réaction acide-base et déterminer la valeur de sa constante thermodynamique d'équilibre en fonction des $pKa$ des couples mis en jeu.
\end{enumerate}

\subsection{Exercice 1 : Vitamine C}

La vitamine C, aussi appelée acide ascorbique, est un diacide noté $AscH_2$.

\begin{enumerate}
	\item Dresser le diagramme de prédominance des espèces acido-basiques issues de l'acide ascorbique en fonction du $pH$ de la solution.
	\item On dissout dans l'eau un comprimé contenant $\SI{500}{mg}$ d'acide ascorbique dans une fiole jaugée de volume $V = \SI{200}{mL}$. Déterminer l'état d'équilibre de la solution obtenue.
	\item La vitamine C existe aussi en comprimé tamponné, réalisée en mélangeant l'acide ascorbique $AscH_2$ et de l'ascorbate de sodium $AscHNa$. Un comprimé de vitamine C tamponnée de masse $m$ en principe actif (acide ascorbique sous ses deux formes, diacide et monoacide) est dissous dans $V'=\SI{100}{mL}$ d'eau distillée. La solution obtenue a un $pH$ égal à $4.4$. Déterminer la masse d'acide ascorbique et la masse d'ascorbate de sodium contenues dans ce cachet. On prendra $m=\SI{500}{mg}$ pour les applications numériques.
\end{enumerate}

\e{Données :} 
\begin{itemize}
	\item Les deux $pKa$ associés à l'espèce étudiée sont $4.2$ et $11.6$.
	\item $M(AscH_2) = \SI{176}{\gram \per \mol}$, $M(AscHNa) = \SI{198}{\gram \per \mol}$.
	\item S'il y a plusieurs réactions possibles, on se concentrera sur celle dont la constante thermodynamique est la plus élevée (réaction prépondérante).
\end{itemize} 

\e{Réponses :}
\begin{enumerate}
	\item -
	\item $[AscH_2] = \SI{1.4e-2}{\mol\per\liter}$, $[AscH_-] = [H_3O^+] = \SI{9.4e-4}{\mol\per\liter}$
	\item $m_a = \SI{1.9e2}{mg}$, $m_b=\SI{3.4e2}{mg}$.
\end{enumerate}

\subsection{Exercice 2 : Titrage du dioxyde de soufre dans le vin}

La concentration en masse de dioxyde de soufre dans un vin blanc ne doit pas excéder $\SI{210}{\milli\gram\per\liter}$. Pour vérifier la conformité de la concentration en dioxyde de soufre d'un vin blanc, on utilise une solution titrante de concentration $C_1 = \SI{7.80e-3}{\mol\per\liter}$ en diiode. Dans un erlenmeyer, on verse un volume $V_2 = \SI{25.0}{mL}$ de vin blanc. On ajoute $\SI{2}{mL}$ d'acide sulfurique pour acidifier le milieu. Lors du titrage du vin blanc, l'équivalence est obtenue après avoir versé un volume $V_E = \SI{6.1}{mL}$ de solution titrante. La réaction support du titrage s'écrit $$\ce{SO2(aq) + I2(aq) + 2H2O(l) \rightarrow SO4^{2-}(aq) + 2I-(aq) + 4H+(aq)}$$

Ce vin est il conforme à la législation ?

\e{Donnée :} $M(SO_2) = \SI{64.1}{g\per\mol}$

\e{Réponse :} $\SI{120}{mg/L} < \SI{210}{mg/L}$.

\subsection{Exercice 3 : Pile à combustible}

Dans certaines piles à combustible, on utilise le dihydrogène comme combustible et le dioxygène comme comburant. 

\begin{enumerate}
	\item Écrire la réaction de combustion du dihydrogène par le dioxygène.
	\item Cette réaction est en fait l'association de deux demi-équations d'oxydoréduction mettant en jeu les couples \ce{H^+/H2_{(g)}} et \ce{O2_{(g)}/H2O}. Écrire les demi-équations d'oxydoréduction.
	\item Les deux demi-réactions ont lieu sur deux électrodes. Indiquer la réaction cathodique et la réaction anodique.
	\item Donner l'expression du potentiel d'oxydoréduction pour les deux couples.
	\item Exprimer la constante d'équilibre $K^o$ en fonction des potentiels standards des couples. Calculer sa valeur. Commenter.
\end{enumerate}

\e{Données :} $E^o(H^+, H_2) = 0.00~V$, $E^o(O_2, H_2O) = 1.23~V$.