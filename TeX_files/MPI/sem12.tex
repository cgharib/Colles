\section{Semaine 12 (16/12-20/12) }


\e{Notions abordées :}
\begin{itemize}
	\item Interférences lumineuses (cf semaine précédente).
	\item Interféromètre de Michelson en lame d'air.
\end{itemize}

\subsection{Exercice 1 : Interférences sur une goutte}

On considère une goutte liquide d'indice $n=\SI{1.4}{}$ sur un solide réfléchissant, éclairée en incidence normale par une source monochromatique de longueur d'onde $\lambda = \SI{546}{nm}$. La goutte est assimilable à une portion de sphère de centre $C$. Son diamètre dans le plan de contact est $d=\SI{0.5}{mm}$ et l'angle de contact à son bord est $\theta = \SI{5}{\degree}$. 

\begin{enumerate}
	\item Justifier que l'on peut négliger la réfraction à la surface de la goutte.
	\item Justifier que des interférences lumineuses apparaissent. Où sont elles localisées ?
	\item Calculer la différence de marche à la surface de la goutte, à une distance $r$ de l'axe de révolution.
	\item Déterminer les rayons des anneaux brillants successifs.
	\item Déterminer l'ordre d'interférence au centre, et au bord. En déduire le nombre d'anneaux visibles.
\end{enumerate}

\e{Réponse :} Au premier ordre en $\theta$, $N = 1 + \lfloor \frac{n d \theta}{2 \lambda}\rfloor = \SI{52}{anneaux}$. (à vérifier)

\subsection{Exercice 2 : Mesure de l'épaisseur d'une lame}

On considère un Michelon en lame d'air éclairé en incidence normale et en lumière blanche, réglé à la teinte plate. En sortie de l'interféromètre, on place un spectromètre connecté à un ordinateur au foyer image d'une lentille convergente.

Dans les bras de l'interféromètre, on place une lame de verre d'indice $n=1.50$ et d'épaisseur $e$.

Sur l'ordinateur, on observe le spectre de la lumière reçue. On remarque que les longueurs d'onde $500.0$, $502.5$, $505.1$, $507.6$, $510.2$ et $512.8$ nm correspondent à des maxima d'intensité.

Expliquer le phénomène observé et en déduire l'épaisseur de la lame.

\e{Réponse :} $e = \SI{0.1}{mm}$.

\subsection{Exercice 3 : Mesure de l'indice de l'air}

Un Michelson est réglé en lame d'air d'épaisseur $e$ et éclairé par une source de lumière quasi-monochromatique de longueur d'onde dans le vide $\lambda_0 = \SI{632}{nm}$. Deux cuves identiques, à faces parallèles et transparentes, de longueur $L=\SI{3.00}{cm}$ sont placées entre la séparatrice et chacun des miroirs. L'écran est placé au foyer image d'une lentille convergente.

Dans la situation initiale, les deux cuves sont pleines d'air à pression atmosphérique. On fait progressivement le vide dans l'une des deux cuves et on voit défiler $26$ franges au centre de l'écran. En déduire l'indice de l'air.

\e{Réponse :} $n_{air} - n_{vide} = \SI{2.7e-4}{}$