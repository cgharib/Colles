\section{Semaine 12 (06/01 - 10/01)}


\e{Notions abordées :}
\begin{itemize}
	\item Interférences lumineuses.
	\item Interféromètre de Michelson.
\end{itemize}

\subsection{Questions de cours :}
\begin{enumerate}
	\item Déterminer l'expression de la différence de marche pour l'interféromètre de Michelson en lame d'air et décrire la figure d'interférences.
	\item Pour l'interféromètre de Michelson en coin d'air, préciser la localisation des interférences. Expliquer comment les observer. Déterminer l'expression de la différence de marche.
	\item Peut on utiliser une source étendue avec le dispositif des trous d'Young ? Et pour le Michelson en configuration lame d'air ? coin d'air ? Conclusion sur l'intérêt de l'interféromètre de Michelson ?
\end{enumerate}

\subsection{Exercices}
cf semaine précédente