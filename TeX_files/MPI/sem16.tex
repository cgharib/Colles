\section{Semaine 16 (27/01-31/01)}


\e{Notions abordées :}
\begin{itemize}
	\item Électrostatique (cf. semaine précédente).
	\item Dipôle électrostatique.
	\item Magnétostatique.
\end{itemize}

\subsection{Questions de cours}
\begin{enumerate}
	\item Montrer que le champ magnétique est à flux conservatif. 
	\item Démontrer l'équation de Maxwell-Ampère à partir du théorème d'Ampère.
	\item Déterminer le champ magnétique généré par un fil rectiligne infini et infiniment fin.
\end{enumerate}


\subsection{Exercice 1 : Bobine torique}

On considère une bobine torique de rayon moyen $R$, comportant $N$ spires, de section carrée de côté $a$ et parcourue par un courant $I$. Déterminer le champ magnétique produit en tout point de l'espace. Comparer avec le champ magnétique du solénoïde infini.

\subsection{Exercice 2 : Caractéristiques d'un câble coaxial}

Un câble coaxial est constitué de deux cylindres conducteurs de rayons $R_1$ et $R_2 > R_1$ sur lesquels circulent, en surface, des courants d'intensité $I$ et de sens opposés.

\begin{enumerate}
	\item Déterminer le champ magnétique.
	\item La densité volumique d'énergie magnétique est donnée par $U_m = \frac{\vec{B}^2}{2 \mu_0}$. En déduire l'inductance linéique du câble coaxial.
	\item Les deux cylindres portent également des charges opposées. Reprendre les calculs pour le champ électrique. En déduire la capacité par unité de longueur.
	\item Quelle relation simple trouve-t-on entre l'inductance et la capacité ?
\end{enumerate}

\subsection{Exercice 3 : Équilibre d'une tige dans un champ non uniforme}

Une tige $[OA]$ de masse $m$ et de longueur $L$ est en rotation autour de l'axe $(Ox)$ horizontal. Son moment d'inertie est $J = \frac{1}{3} m L^2$. Elle est parcourue par un courant d'intensité $i$ dirigée de $O$ vers $A$. Son inclinaison par rapport à la verticale est mesurée par l'angle $\theta$. Un fil rectiligne d'axe vertical $(Oz)$ est parcouru par un courant de même intensité $i$ dirigée vers le haut. À l'équilibre de la tige, donner l'expression de $i$ en fonction de $\theta$, $L$, $m$ et $g$. 


\subsection{Exercice 4 : Action mécanique d'un fil sur un autre fil parallèle}

Deux fils rectilignes infinis parallèles sont distants de $d = \SI{1.0}{m}$. Ils sont parcourus par des courants de même intensité $I$ et de même sens. La force subie par un tronçon de longueur $L = \SI{1.0}{m}$ d’un des fils est égale $\SI{2.0e-7}{N}$.

Déterminer la valeur de $I$.