\section{Semaine 06 (04/11-08/11) }


\e{Notions abordées :}
\begin{itemize}
	\item Mécanique de MPSI.
	\item Dynamique en référentiel non galiléen.
	\item Lois du frottement de Coulomb.
\end{itemize}

\subsection{Exercice 1 : Glissement d'une caisse dans un camion}

\begin{minipage}[c]{\linewidth/2}
	\includegraphics{./Images/mp_s05_ex01.png}
\end{minipage}%
\begin{minipage}[c]{\linewidth/2}
	Le camion accélère avec l'accélération constante $a$.
	\begin{enumerate}
		\item À quelle condition le glissement commence-t-il ?
		\item Au bout de combien de temps la caisse atteint-elle le rebord ?
		\item Quelle distance parcourt-elle après être tombée ?
		\item La caisse glisse-t-elle ou bascule-t-elle lors de l'accélération ?
	\end{enumerate}
\end{minipage}

\subsection{Exercice 2 : Cube sur un plan incliné}

Un cube repose sur un plan incliné d'un angle $\alpha$. On augmente $\alpha$ très lentement.

\begin{enumerate}
	\item À quelle condition le glissement commence-t-il ?
	\item À quelle condition le cube bascule-t-il ?
	\item Qu'est ce qui arrive en premier ? On donne le coefficient de frottement bois-bois $f = 0.4$.
\end{enumerate}

\subsection{Exercice 3 : Glissement et liaison avec une corde}

\begin{minipage}[c]{\linewidth/2}
	\includegraphics{./Images/mp_s05_ex03.png}
\end{minipage}%
\begin{minipage}[c]{\linewidth/2}
	Deux caisses sont liées par une corde qui passe par une poulie. On prend en compte le frottement de la grosse caisse sur la surface. En précisant les hypothèses utilisées, déterminer l'altitude de la caisse suspendue en fonction du temps.
\end{minipage}