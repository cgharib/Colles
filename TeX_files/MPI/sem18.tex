\section{Semaine 18 (10/02-14/02)}


\e{Notions abordées :}
\begin{itemize}
	\item Équations de Maxwell.
	\item Ondes électromagnétiques dans le vide.
	\item Réflexion sur un conducteur parfait.
\end{itemize}

\subsection{Questions de cours}
\begin{enumerate}
	\item Déterminer l'équation de propagation des ondes électromagnétiques dans le vide.
	\item Déterminer l'équation de dispersion des OPPH électromagnétiques dans le vide. Définir vitesse de phase et vitesse de groupe, les calculer.
	\item Démontrer la relation de structure pour une OPPH électromagnétique. Cas du vide.
\end{enumerate}

\subsection{Exercice 1 : Propagation guidée entre deux plans}

Deux plans infinis conducteurs parfaits délimitent une cavité vide entre $z=0$ et $z=b$. 

\begin{enumerate}
	\item Rappeler l'équation de propagation du champ électromagnétique dans la cavité.
	\item Justifier que, contrairement à ce qu'on fait d'habitude, on ne peut pas chercher une solution polarisée suivant $\vec{u_y}$ sous la forme d'une OPPH.
	
	On cherche une solution de l'équation sous la forme $$\vec{E}(x, z, t) = \beta(z)\vec{u_y}\cos(\omega t - k x)$$.
	
	\item Déterminer l'équation différentielle vérifiée par $\beta$. En donner la solution générale.
	\item Montrer que l'onde ne peut exister que si $\omega > kc$. Commenter en relation avec la relation de dispersion habituelle.
	\item Déterminer complètement $\beta(z)$.
	\item Déterminer la relation de dispersion. 
	\item En déduire qu'il existe une pulsation minimale pour qu'une onde électromagnétique se propage dans le guide.
\end{enumerate}

\e{Réponse :} $\vec{k}^2 = \frac{\omega^2}{c^2} - \frac{n^2 \pi^2}{b^2}$

\subsection{Exercice 2 : Lois de Descartes sur la réflexion et la réfraction}

L'objectif de cet exercice est de redémontrer les lois de Descartes à l'interface entre deux diélectriques.

On rappelle que dans un diélectrique, tout se passe comme dans le vide, à condition de remplacer $\epsilon_0$ par $\epsilon = \epsilon_0 \epsilon_r$.

On considère donc deux diélectriques accolés. Le milieu $1$ de permittivité $epsilon_1$ occupe le demi-espace $z<0$. Le milieu $2$ de permittivité $epsilon_2$ occupe le demi-espace $z>0$.

On considère également une OPPH électromagnétique incidente $(\vec{E_i}, \vec{B_i}, \omega_i, \vec{k_i} = k_{i,x}\vec{e_x} + k_{i, z}\vec{e_z})$ provenant du milieu $1$ vers l'interface avec un angle $\alpha$ par rapport à la normale. Pour simplifier, on suppose le champ électrique incident polarisé rectilignement selon la perpendiculaire au plan d'incidence.

On rappelle qu'à une interface, la composante tangentielle du champ électrique est continue.

\begin{enumerate}
	\item Rappeler les trois lois de Descartes.
	
	\item Justifier qu'il doit exister une onde transmise et/ou une onde réflechie. 
	
	On appellera $(\vec{E_t}, \vec{B_t}, \omega_t, \vec{k_t})$ l'onde transmise et $(\vec{E_r}, \vec{B_r}, \omega_r, \vec{k_r})$ l'onde réfléchie. 
	
	\item Justifier qualitativement que, si ils existent, les champ électrique transmis et réfléchi ont la même polarisation que le champ électrique incident.
	
	\item Sur un schéma, faire figurer l'interface, les trois ondes ainsi que les angles respectifs.
	
	\item Montrer que les ondes ont toute la même pulsation.
	
	\item Montrer que les ondes ont toute le même $k_x$.
	
	\item Montrer que $\frac{\vec{k_t}^2}{\epsilon_{r2}} = \frac{\vec{k_r}^2}{\epsilon_{r1}} = \frac{\vec{k_i}^2}{\epsilon_{r1}}$.
	
	\item En déduire les lois de Descartes sur la réflexion et la réfraction.
\end{enumerate}

\subsection{Exercice 3 : Équation de Klein-Gordon et masse du photon}

\begin{enumerate}
	\item Rappeler l'équation de propagation des ondes électromagnétiques dans le vide.
	
	Dans le cadre de la théorie électromagnétique étendue au cas d'un photon de masse non nulle, l'équation de propagation du champ électrique devient $$\Delta \vec{E} - \frac{1}{c^2} \frac{\partial^2 \vec{E}}{\partial t^2} = \eta^2 \vec{E}$$
	
	\item Quelle est l'unité de $\eta$ ?
	\item Déterminer la relation de dispersion.
	\item Exprimer les vitesses de phase et de groupe en fonction de $c$, $\omega$ et $\eta$. Les tracer. Commentaires ?
	\item Rappeler les expressions de l'énergie $E$ et de la quantité de mouvement $\vec{p}$ d'un photon.
	\item Sachant que pour une particule relativiste de masse $m$, on a $E^2 = p^2 c^2 + m^2 c^4$, exprimer la masse du photon en fonction de $\eta$, $\hbar$ et $c$.
	
	Deux photons de longueurs d'ondes $\lambda_1$ et $\lambda_2$ sont émis au même instant par une source ponctuelle située à une distance $L$. On supposera $\eta^2 \lambda_{1, 2}^2 \ll 1$.
	
	\item Exprimer la différence $\delta t$ des temps de réception des deux signaux.
	
	\item L'observation de certaines étoiles doubles donne $\delta t < \SI{1e-3}{s}$ pour $\lambda_1 = \SI{0.4}{\micro\meter}$, $\lambda_2 = \SI{0.8}{\micro\meter}$ et $L = \SI{1e3}{\textrm{années lumières}}$. En déduire une limite supérieure pour la masse du photon. Commentaire.
	
\end{enumerate}