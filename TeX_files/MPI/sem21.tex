\section{Semaine 21 (17/03-21/03)}


\e{Notions abordées :}
\begin{itemize}
	\item Rayonnement thermique.
	\item Conduction thermique.
\end{itemize}

\subsection{Questions de cours}
\begin{enumerate}
	\item Définir le vecteur densité de courant de chaleur. Donner la loi de Fourier.
	\item Définir la résistance thermique. 
	\item Donner la loi de Newton sur le transfert conducto-convectif.
\end{enumerate}


\subsection{Exercice 1 : Résolution de l'équation de la chaleur en régime permanent dans un barreau cylindrique}

Un barreau solide indéformable de conductivité thermique $\lambda$ a la forme d'un cylindre de longueur $L$ et de section $S$. Il est calorifugé sur sa paroi latérale et au contact d'un thermostat à la température $T_0$ en $x = 0$.

\begin{enumerate}
	\item Établir l'équation différentielle vérifiée par $T(x, t)$. 
	
	Dans la suite, on se place en régime permanent.
	
	\item En déduire l'expression de la température en fonction de deux constantes $A$ et $B$.

	\item En $x = L$, on impose successivement plusieurs contraintes thermiques. Déterminer dans chaque cas les constantes $A$ et $B$. En déduire le flux thermique traversant le barreau.
	\begin{enumerate}
		\item Le bout du barreau est au contact d'un thermostat à la température $T_1$.
		\item Le bout du barreau est au contact d'un fluide à la température $T_1$.
		\item Le bout du barreau est dans le vide spatial.
	\end{enumerate}
\end{enumerate}

\e{Données :}
\begin{itemize}
	\item $L = \SI{1.0}{m}$ et $S = \SI{1.0}{\meter\squared}$.
	\item $\lambda = \SI{0.567}{\per\watt\per\kelvin\per\meter}$ et  $h = \SI{15}{\watt\per\meter\squared\per\kelvin}$
	\item $T_0 = \SI{300}{K}$ et $T_1 = \SI{2.0}{\degreeCelsius}$
	\item $\sigma = \SI{5.67e-8}{\watt\per\meter\squared\per\kelvin\tothe{4}}$
\end{itemize}

\subsection{Exercice 2 : Ondes thermiques}

Un barreau solide indéformable de conductivité thermique $\lambda$ a la forme d'un cylindre de longueur $L$ et de section $S$. Il est calorifugé sur sa paroi latérale et au contact d'un thermostat à la température $T_0 = \Theta_0 \cos(\omega t)$ en $x = 0$.

\begin{enumerate}
	\item Établir l'équation différentielle vérifiée par $T(x, t)$.
	
	Vu la condition initiale, on cherche une solution de l'équation sous la forme d'une onde plane. 
	
	\item Écrire la forme de la solution recherchée en prenant en compte la condition au bord.
	\item Déterminer la relation de dispersion.
	\item En écrivant la solution en onde plane, déterminer une distance caractéristique d'atténuation $\delta$.
	\item Application numérique. On donne $\rho = \SI{1000}{\kilogram\per\meter\cubed}$, $c = \SI{4200}{\joule\per\kelvin\per\kilogram}$ et $\lambda = \SI{0.500}{\watt\per\meter\per\kelvin}$.
	\begin{enumerate}
		\item L'excitation thermique journalière pendant l'été a pour valeur moyenne $\SI{25}{\degreeCelsius}$ et une amplitude de $\SI{10}{\degreeCelsius}$. Donner la profondeur à partir de laquelle on peut considérer la variation de température comme négligeable.
		
		\item L'excitation thermique annuelle a pour valeur moyenne $\SI{15}{\degreeCelsius}$ et une amplitude de $\SI{10}{\degreeCelsius}$. Donner la profondeur à partir de laquelle on peut considérer la variation de température comme négligeable. Pourquoi parle-t-on dans ce cas de "théorie des caves" ?
	\end{enumerate}
\end{enumerate}

\e{Réponses :}
\begin{enumerate}
	\item -
	\item -
	\item -
	\item -
	\item -
	\begin{enumerate}
		\item $\delta = \SI{5.7}{\centi\meter}$
		\item $\delta = \SI{1.1}{\meter}$
	\end{enumerate}
\end{enumerate}


\subsection{Exercice 3 : Isolation extérieure ou intérieure}

Un mur de surface $S = \SI{1.0}{\meter\squared}$, d'épaisseur $E = \SI{30}{\cm}$, de conductivité thermique $\Lambda = \SI{1}{\watt\per\meter\per\kelvin}$, de capacité thermique massique $C = \SI{500}{\joule\per\kelvin\per\kilogram}$ et de masse volumique $\mu = \SI{2000}{\kg\per\cubic\meter}$ est isolé avec un isolant thermique de même surface $S$, d'épaisseur $e = E/3$, de conductivité thermique $\lambda = \Lambda/10$, de capacité thermique massique $c = C/2$ et de masse volumique $\rho = M/10$. La température intérieure à la maison est $T_i = \SI{20}{\degreeCelsius}$, la température extérieure est $T_e = \SI{10}{\degreeCelsius}$.

\begin{enumerate}
	\item Établir l'équation vérifiée par $T(x, t)$.
	\item Calculer les résistances thermiques $R$ du mur et $r$ de l'isolant. Commentaire.
	\item Justifier qu'en régime permanent la position de l'isolant (sur la face intérieure ou extérieure du mur) ne change pas la valeur du flux thermique allant de l'intérieur vers l'extérieur. 
	\item Déterminer la puissance que doit fournir un radiateur pour maintenir la température intérieure à $\SI{20}{\degreeCelsius}$ sans et avec isolant. Commentaire.
	\item Calculer la température $T_I$ au point d'interface entre le mur et l'isolant dans chaque cas.
	
	À $t=0$, la température du mur et de l'isolant est uniforme et égale à $T_e$. 
	\item Déterminer l'ordre de grandeur de la durée du régime transitoire.
	
	\item Déterminer l'énergie thermique reçue par le mur et l'isolant entre l'état initial et le régime permanent dans les deux cas, en effectuant un bilan. Commentaire.
\end{enumerate}

\subsection{Exercice 4 : Température d'équilibre de la Terre}

On cherche à déterminer la température moyenne de la Terre.

\begin{enumerate}
	\item On néglige d'abord la présence de l'atmosphère. Grâce à un bilan radiatif, déterminer la température de surface de la Terre en fonction de la distance terre-soleil $d = \SI{149e9}{\meter}$, de la puissance totale émise par le soleil $P_{sol} = \SI{3.8e26}{\watt}$ et de la constante de Stefan $\sigma = \SI{5.67e-8}{\watt\per\meter\squared\per\kelvin\tothe{4}}$. Application numérique. Commentaire.
	\item On considère maintenant l'influence de l'atmosphère. Qualitativement, quelle propriété de l'atmosphère explique l'effet de serre ?
	\item En réalisant un bilan sur le système {terrre} et le système {terre+atmosphere}, déterminer la température de surface de la Terre. Commentaire.
	\item En fait, l'atmosphère n'est pas totalement transparente au rayonnement solaire. Elle en absorbe une partie $\alpha = 0.33$. En déduire la température de surface de la Terre. Commentaire.
\end{enumerate}

\e{Réponses :}
\begin{enumerate}
	\item $\SI{-18}{\degreeCelsius}$
	\item $\SI{30}{\degreeCelsius}$
	\item $\SI{17}{\degreeCelsius}$
\end{enumerate}