\section{Semaine 15 (20/01-24/01)}


\e{Notions abordées :}
\begin{itemize}
	\item Électrostatique.
\end{itemize}

\subsection{Questions de cours}
\begin{enumerate}
	\item Démontrer, à partir du théorème de Gauss, l'équation de Maxwell-Gauss.
	\item Exprimer, à partir du théorème de Gauss, le champ électrique généré par une charge ponctuelle $q$.
	\item Exprimer la capacité d'un condensateur plan.
\end{enumerate}

\subsection{Exercice 1 : Condensateur cylindrique}

On considère deux électrodes cylindriques infinies, de même axe, de rayons $R_1$ et $R_2 = R_1 + e$. Le cylindre central porte la charge surfacique $+\sigma$.

\begin{enumerate}
	\item Quelle est la charge surfacique portée par le cylindre extérieur ?
	\item Sur un schéma, faire figurer les électrodes, leurs dimensions et le champ électrique.
	\item Déterminer l'expression de la capacité d'un tronçon de longueur $H$ du condensateur en fonction des paramètres géométriques et de $\epsilon_0$.
	\item Commenter le cas $e \ll R_1$.
\end{enumerate}

\subsection{Exercice 2 : Plaque épaisse chargée}

On considère une plaque épaisse de largeur et longueur $L$ occupant l'espace entre $z=-a/2$ et $z=+a/2$. Elle porte une charge $Q$ répartie uniformément.

On réalise l'étude au voisinage de $x = 0$ et $y = 0$, ce qui permet de négliger les effets de bords. 

\begin{enumerate}
	\item Déterminer le champ électrique qu'elle génère.
	\item Dans la limite $a \longrightarrow 0$ :
	\begin{enumerate}
		\item Déterminer la charge surfacique $\sigma$.
		\item Déterminer la relation de passage entre $\overrightarrow{E(z=0^+)}$ et $\overrightarrow{E(z=0^-)}$.
	\end{enumerate}
\end{enumerate}

\subsection{Exercice 3 : Condensateur à cylindres parallèles}

On considère deux cylindres infinis identiques, de rayon $a$. Les axes des deux cylindres sont parallèles et distants de $2 D$. Le cylindre $1$ porte la charge surfacique $+\sigma$.

\begin{enumerate}
	\item Quelle est la charge surfacique portée par le deuxième cylindre ?
	\item Sur un schéma, faire figurer les cylindres, leurs dimensions, les charges et le champ électrique.
	\item À l'aide du théorème de superposition, déterminer l'expression de la capacité d'un tronçon de longueur $H$ du condensateur en fonction des paramètres géométriques et de $\epsilon_0$.
\end{enumerate}
