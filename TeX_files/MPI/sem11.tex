\section{Semaine 11 (09/12-13/12) }


\e{Notions abordées :}
\begin{itemize}
	\item Interférences lumineuses.
	\item Sources élargies spatialement et spectralement.
	\item (Pas de Michelson).
\end{itemize}

\subsection{Exercice 1 : Cohérence spatiale des fentes d'Young}

Quelle est la largeur d'une source étendue qui provoque un brouillage de la figure d'interférence dans le cas des fentes d'Young ?

\subsection{Exercice 2 : Interférences sur une goutte}

On considère une goutte liquide d'indice $n=\SI{1.4}{}$ sur un solide réfléchissant, éclairée en incidence normale par une source monochromatique de longueur d'onde $\lambda = \SI{546}{nm}$. La goutte est assimilable à une portion de sphère de centre $C$. Son diamètre dans le plan de contact est $d=\SI{0.5}{mm}$ et l'angle de contact à son bord est $\theta = \SI{5}{\degree}$. 

\begin{enumerate}
	\item Justifier que l'on peut négliger la réfraction à la surface de la goutte.
	\item Justifier que des interférences lumineuses apparaissent. Où sont elles localisées ?
	\item Calculer la différence de marche à la surface de la goutte, à une distance $r$ de l'axe de révolution.
	\item Déterminer les rayons des anneaux brillants successifs.
	\item Déterminer l'ordre d'interférence au centre, et au bord. En déduire le nombre d'anneaux visibles.
\end{enumerate}

\e{Réponse :} Au premier ordre en $\theta$, $N = 1 + \lfloor \frac{n d \theta}{2 \lambda}\rfloor = \SI{52}{anneaux}$. (à vérifier)

\subsection{Exercice 3 : Mesure de l'indice d'un verre}

On considère le dispositif des fentes d'Young avec lentille éclairé par une source ponctuelle monochromatique au foyer objet d'une lentille convergente. Devant l'une des deux fentes, on place une lame de verre d'épaisseur $e$ connue et d'indice $n$ inconnu.

\begin{enumerate}
	\item Décrire la figure d'interférences. Peut-on déterminer l'indice du verre ?
	\item Montrer qualitativement que l'on peut déterminer l'indice du verre en passant en lumière blanche.
	\item Le montrer quantitativement en considérant une source lumineuse de distribution spectrale d'intensité $\frac{dI}{d\nu}$ homogène, de fréquence centrale $\nu_0$ et de largeur $\Delta \nu$.
\end{enumerate}