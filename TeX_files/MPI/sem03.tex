\section{Semaine 03 (30/09-04/10) }


\e{Notions abordées :}
\begin{itemize}
	\item Électronique de MPSI.
	\item Filtrage d'un signal périodique.
	\item Numérisation.
	\item Portes logiques.
\end{itemize}

\subsection{Exercice 1 : Intégration d'un créneau par un filtre passe bande}

Une tension créneau est injectée dans un filtre passe-bande non inverseur d'ordre 2, de pulsation de résonance $\omega_0$, de facteur de qualité $Q$ et de gain maximum $G_0$. La pulsation $\omega$ de la tension créneau est supposée grande devant $\omega_0$.

\begin{enumerate}
	\item Écrire la fonction de transfert du filtre.
	\item Montrer que ce filtre se comporte vis-à-vis du créneau d'entrée comme un intégrateur.
	\item Écrire l'équation différentielle reliant la tension d'entrée $v_e(t)$ et la tension de sortie $v_s(t)$ de l'intégrateur. Qu'obtient-on précisément en sortie du filtre ? Comment seraient modifiés les résultats si on ajoutait une tension continue au créneau à l'entrée ?
\end{enumerate}

\subsection{Exercice 2 : Shannon comme au cinéma}

Au cinéma, lorsqu'on regarde les roues d'une voiture qui démarre, on les voit d'abord tourner dans le sens réel puis elles semblent tourner à l'envers. Expliquer d'où provient cette illusion. Qu'observe-t-on en visionnant le film lorsque les roues de la voiture tournent à $f_1=\SI{1200}{tours/min}$ ? Et à $f_2 = \SI{1680}{tours/min}$.

