\section{Semaine 17 (03/02-07/02)}


\e{Notions abordées :}
\begin{itemize}
	\item Magnétostatique.
	\item Dipôle magnétique.
	\item Équations de Maxwell.
	\item Révisions d'induction.
\end{itemize}

\subsection{Questions de cours}
\begin{enumerate}
	\item Montrer que le champ magnétique est à flux conservatif. 
	\item Démontrer l'équation de Maxwell-Ampère à partir du théorème d'Ampère.
	\item Donner les équations de Maxwell.
\end{enumerate}


\subsection{Exercice 1 : Un exo d'induction}

\subsection{Exercice 2 : Inductances propres et mutuelles}

\begin{enumerate}
	\item Rappeler la définitions des inductances propres et mutuelles. À quoi ces grandeurs nous servent-elles ?
	\item Calculer l'inductance propre d'un solénoïde de longueur $l$, de section $S$, comportant $N$ spires, et supposé suffisamment long pour négliger les effets de bords. Application numérique pour une bobine de TP.
	\item Exprimer l'énergie magnétique stockée dans la bobine. Retrouver l'expression de la densité volumique d'énergie magnétique.
	\item On considère un petit solénoïde à l'intérieur d'un gros solénoïde, les deux partageant le même axe. Déterminer l'inductance mutuelle.
\end{enumerate}

\e{Réponses :}
\begin{enumerate}
	\item -
	\item $L = \mu_0 \frac{N^2 S}{l}$
	\item -
	\item $M = \mu_0 \frac{N_1 N_2}{l_1}S_1$
\end{enumerate}

\subsection{Exercice 3 : Induction par un aimant mobile}

Une spire circulaire d'axe $(Oz)$, de rayon $a$ et de résistance $R$ est immobile. Sur son axe, on rapproche à vitesse $\vec{V} = V \vec{e_z}$ constante un petit aimant assimilé à un dipôle magnétique de moment $\vec{\mathcal{M}} = \mathcal{M}\vec{e_z}$.

\begin{enumerate}
	\item Prévoir le sens du courant $i$ induit dans la spire.
	\item Calculer le courant $i$ dans la spire.
	\item Quelle est la force exercée par l'aimant sur la spire ? Commenter.
\end{enumerate}