\section{Semaine 03 (30/09-04/10) }

\e{Notions abordées :}
\begin{itemize}
	\item Traitement numérique du signal.
	\item Mécanique de MPSI.
	\item Dynamique en référentiel non galiléen.
\end{itemize}

\subsection{Exercice 1}

Une tige rigide est en rotation uniforme autour de son axe à la pulsation $\omega$. Un mobile M est lié par un fil au point O situé sur l'axe à l'altitude $h$. 

\begin{enumerate}
	\item Démontrer la loi de composition des accélérations pour un référentiel en rotation uniforme.
	\item Déterminer l'angle $\alpha_0$ d'équilibre du mobile.
	\item Étudier la stabilité de la position d'équilibre.
\end{enumerate}

\subsection{Exercice 2}

Un électron et un proton de même énergie cinétique sont plongés dans un champ magnétique uniforme, orthogonal à leur vitesse initiale.

\begin{enumerate}
	\item Décrire qualitativement les trajectoires.
	\item Comparer :
	\begin{itemize}
		\item Leur vitesse.
		\item Le rayon de leur trajectoire.
		\item Leur période.
	\end{itemize}
	\item Calculer la force centrifuge subie par l'électron.
\end{enumerate}

\subsection{Exercice 3}

Un mobile $M$ coulisse sans frottement  sur un axe horizontal $(Ox)$ dans un train qui accélère avec une accélération $A\vec{u_x}$, le point $O$ étant fixé à l'arrière du wagon. Entre $O$ et $M$ on place un ressort $(k, l_0)$. À $t=0$, $x=l_0$ et la vitesse de $M$ dans le référentiel du train est nulle.

\begin{enumerate}
	\item Démontrer la loi de composition des accélérations dans un référentiel uniformément accéléré.
	\item Établir $x(t)$.
\end{enumerate}



