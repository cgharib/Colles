\section{Semaine 06 (04/11-08/11) }

\e{Notions abordées :}
\begin{itemize}
	\item Particules dans un $\vec{E}, \vec{B}$ statique (MPSI).
	\item Électrostatique :
	\begin{itemize}
		\item Distributions de charges et de courants.
		\item Symétries et invariances.
		\item Loi de Coulomb.
		\item Théorème de Gauss.
		\item Analogie gravitationnelle.
	\end{itemize}
\end{itemize}

\subsection{Exercice 1 : Condensateur cylindrique}

Deux cylindres métalliques $\mathcal{C}_1$ et $\mathcal{C}_2$ de même axe $(Oz)$, de même hauteur $h$ et de rayon $R_1$ et $R_2 > R_1$ portent des charges réparties uniformément en surface. On note $\sigma_1$ la densité surfacique de charge de $\mathcal{C}_1$.

\begin{enumerate}
	\item Quelle est la charge portée par $\mathcal{C}_2$ ? En déduire sa densité surfacique de charges.
	\item Déterminer la capacité $C$ de ce condensateur cylindrique. 
	\item Dans quel cas retrouve-t-on la capacité d'un condensateur plan ?
\end{enumerate}

\e{Réponse :} $C = \frac{2 \pi \epsilon_0 h}{\ln{R_2/R_1}}$

\subsection{Exercice 2 : Condensateur sphérique}

Deux sphères métalliques $\mathcal{S}_1$ et $\mathcal{S}_2$ de même centre $O$ et de rayons $R_1$ et $R_2 > R_1$ portent des charges réparties uniformément en surface. On note $\sigma_1$ la densité surfacique de charge de $\mathcal{S}_1$.

\begin{enumerate}
	\item Quelle est la charge portée par $\mathcal{S}_2$ ? En déduire sa densité surfacique de charges.
	\item Déterminer la capacité $C$ de ce condensateur sphérique.
	\item Dans quel cas retrouve-t-on la capacité d'un condensateur plan ?
\end{enumerate}

\e{Réponse :} $C = \frac{4 \pi \epsilon_0}{\frac{1}{R_1}-\frac{1}{R_2}}.$

\subsection{Exercice 3 : Rayon classique de l'électron}

L’électron de charge $-e$ est modélisé par une sphère $\mathcal{S}$ de centre $O$ et de rayon $R$ uniformément chargée dans son volume.

\begin{enumerate}
	\item Déterminer le champ électrique généré par l'électron.
	\item Évaluer l'énergie électrique $U_e$ d'un électron isolé liée à la seule présence du champ électrostatique qu'il crée.
	\item En assimilant cette énergie à l'énergie de repos $E=mc^2$ prévue par la relativité, déterminer le rayon $R_e$ de l'électron. Commentaire.
\end{enumerate}

\e{Réponse :} $R_e = \frac{3e^2}{20 \pi \epsilon_0 m_e c^2} = \SI{1.7e-15}{\meter}$