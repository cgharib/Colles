\section{Semaine 02 (23/09-27/09) }


\e{Notions abordées :}
\begin{itemize}
	\item Mécanique du point.
	\item Traitement numérique du signal.
\end{itemize}

\subsection{Exercice 1}

\begin{enumerate}
	\item Définir un satellite géostationnaire et calculer son altitude.
	\item Quel travail faut il fournir pour augmenter son altitude de $\SI{50}{km}$.
\end{enumerate}

\subsection{Exercice 2}

On considère un point matériel astreint à se déplacer autour d'un anneau en rotation autour d'un diamètre, à $\omega$ constante.

Positions d'équilibre ? Stabilité ?

\subsection{Exercice 3}

On cherche à graver sur un $CD$ une musique. Toutefois, il existe un signal parasite à $f_p = \SI{42.1}{kHz}$. 

\begin{enumerate}
	\item Échantillonnage sur 16 bits. Quelle est la taille du fichier si la durée vaut $74$ minutes.
	\item Le critère de Shannon est il vérifié ? Conséquence ?
	\item Comment résoudre ce problème ?
\end{enumerate}


\subsection{Exercice 4}

Décrire le mouvement d'une particule chargée dans un champ magnétique statique uniforme.
