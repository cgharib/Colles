\section{Semaine 10 (02/12-06/12) }

\e{Notions abordées :}
\begin{itemize}
	\item Application du premier principe à la transformation chimique. (cf. semaine précédente)
	\item Application du second principe à la transformation chimique.
\end{itemize}

\subsection{Questions de cours}

\begin{enumerate}
	\item Montrer, en précisant les conditions expérimentales, que l'enthalpie libre est un potentiel thermodynamique. En déduire le critère d'évolution des transformations chimiques.
	\item Quel est le lien entre la constante de réaction et l'enthalpie standard de réaction ? Démontrer, dans le cadre de l'approximation d'Ellingham, la loi de Van't Hoff sur l'évolution de $K^\circ$ en fonction de la température.
	\item Définir le potentiel chimique d'un constituant dans un mélange. Quel est l'intérêt de cette quantité pour déterminer l'équilibre chimique ? Donner son expression dans des cas usuels.
\end{enumerate}

\subsection{Exercice 1 : Interprétation des enthalpies et entropies de réaction}

Dans une réaction de grillage, le sulfure du plomb réagit avec le dioxygène pour former de l'oxyde de plomb et du dioxyde de soufre.

\begin{enumerate}
	\item Calculer l'enthalpie standard de réaction à $\SI{298}{K}$. Commentaire ?
	\item Calculer l'enthalpie standard de réaction à $\SI{1223}{K}$. Commentaire ?
	\item Calculer l'entropie standard de réaction à $\SI{298}{K}$. Commentaire ?
	\item Calculer la constante de réaction à $\SI{298}{K}$. Commentaire ?
\end{enumerate}

\e{Données :}

\vspace{5mm}

\begin{tabular}{|c|c|c|c|c|}
	\hline
	& \ce{PbS_{(s)}} & \ce{PbO_{(s)}} & \ce{O2_{(g)}} & \ce{SO2_{(g)}} \\ \hline
	$\Delta_fH^\circ(\SI{298}{K})~(\SI{}{\kilo\joule\per\mole})$ & $-100.4$ & $-217.4$ & & $-296.8$ \\ \hline
	$c_p^\circ~(\SI{}{\joule\per\kelvin\per\mole})$ & 49.5 & 45.8 & 29.4 & 39.9 \\ \hline
	$S^\circ_{mol}~(\SI{}{\joule\per\kelvin\per\mole}$ & 91.2 & 68.7 & 205.1 & 248.2 \\ \hline
\end{tabular}

\vspace{5mm}

\e{Réponses :}
\begin{enumerate}
	\item $\SI{-413.8}{\kilo\joule\per\mole}$
	\item $\SI{-406.5}{\kilo\joule\per\mole}$
	\item $\SI{-82.0}{\joule\per\kelvin\per\mole}$
	\item $\SI{1.5e68}{}$
\end{enumerate}

\subsection{Exercice 2 : Étude du produit ionique}

On relève le $pH$ de l'eau à différentes températures.

\vspace{5mm}

\begin{tabular}{|c|c|c|c|c|c|}
	\hline
	$T~(\SI{}{\degree C})$ & 0 & 18 & 25 & 50 & 100 \\ \hline
	$pH$ & 7.47 & 7.12 & 7.00 & 6.63 & 6.12 \\ \hline
\end{tabular}

\vspace{5mm}

\begin{enumerate}
	\item Quel est l'équilibre à considérer, et qui est responsable de la variation du $pH$ en fonction de la température ?
	\item Déterminer son enthalpie et son entropie de réaction. Commenter.
	\item En déduire l'expression du produit ionique de l'eau en fonction de la température.
\end{enumerate}

\e{Réponses :} Enthalpie : $\SI{51.9}{\kilo\joule\per\mole}$, entropie : $\SI{94.5}{\joule\per\kelvin\per\mole}$

\subsection{Exercice 3 : Existence du diamant}

\begin{enumerate}
	\item Rappeler l'expression du potentiel chimique pour une phase condensée.
	\item Montrer qu'à $298$ K, il ne peut exister de carbone diamant en équilibre avec le carbone graphite. Comment expliquer alors l'existence du diamant dans la vie "quotidienne" ?
	\item Rappeler la différentielle de l'enthalpie libre d'un corps pur ainsi que l'expression de l'enthalpie libre d'un corps pur en fonction de la quantité de matière et du potentiel chimique.
	\item En déduire la différentielle du potentiel chimique d'un corps pur, puis une expression du potentiel chimique d'un corps pur à pression $P$ en fonction du potentiel chimique à $P^\circ$ et du volume molaire.
	\item Pour une phase condensée idéale, que peut on dire du volume molaire ? En déduire une expression simplifiée du potentiel chimique.
	\item En déduire l'expression de l'enthalpie libre de réaction en fonction de l'enthalpie libre standard, des volumes molaire et de la pression.
	\item Déterminer la pression minimale pour obtenir du diamant.
\end{enumerate}

\e{Réponses :}
\begin{enumerate}
	\item
	\item $\Delta_R G^0 = \SI{2.8}{\kilo\joule\per\mole}$.
	\item 
	\item 
	\item 
	\item
	\item $\SI{1.5}{\giga\pascal}$
	
\end{enumerate}