\section{Semaine 15 (20/01-24/01) }

\e{Notions abordées :}
\begin{itemize}
	\item Ondes électromagnétiques dans le vide.
	\item Ondes électromagnétiques dans les milieux.
	\item Effet de peau.
\end{itemize}

\subsection{Exercice 1 : Chauffage d'un métal par le soleil}

Un métal, modélisé par un conducteur ohmique non chargé, de conductivité $\gamma = \SI{38}{\mega S\per\meter}$ occupe le demi-espace $z \geq 0$. 

\begin{enumerate}
	\item Écrire les équations de Maxwell dans le métal.
	\item On s'intéresse à des ondes électromagnétiques de fréquences inférieures ou égales à celles du visible. Par une analyse d'ordres de grandeur, montrer que l'on peut négliger le courant de déplacement dans l'équation de Maxwell-Ampère.
	\item En déduire l'équation de propagation des ondes électromagnétiques dans un conducteur.
	\item Établir l'équation de dispersion. En déduire l'épaisseur de peau $\delta$ en fonction de $\omega$.
	\item Écrire, en notation complexe, le champ électrique dans le conducteur pour une onde polarisée selon $\vec{e_x}$, se propageant selon $e_z$ d'amplitude $E_0$ et de pulsation $\omega$.
	\item Exprimer le champ magnétique associé à cette onde.
	\item Déterminer la puissance volumique dissipée par effet Joule dans le conducteur. Calculer sa valeur moyenne. 
	\item En déduire la puissance totale dissipée dans une portion $\mathcal{V} = [0, a]\times[0, a]\times[0, +\infty[$ de conducteur. 
	\item Pourquoi dit-on que ce sont les infrarouges qui font chauffer une carrosserie de voiture exposée au soleil ?
\end{enumerate}

\subsection{Exercice 2 : Ondes électromagnétiques dans un diélectrique}

Dans un milieu diélectrique, les équations de Maxwell sont identiques à celles dans le vide, à condition de remplacer $\epsilon_0$ par $\epsilon_0 \epsilon_r$ avec $\epsilon_r > 1$ la permittivité diélectrique relative du milieu.

\begin{enumerate}
	\item Écrire les équations de Maxwell dans un diélectrique non chargé.
	\item Déterminer l'équation de propagation d'une onde électromagnétique dans un milieu diélectrique.
	\item Déterminer l'équation de dispersion.
	\item Exprimer les vitesses de phase et de groupe. 
	\item En déduire l'expression de l'indice de réfraction en fonction de $\epsilon_r$.
	
	En fait, un modèle microscopique (électron élastiquement lié) donne une permittivité relative complexe $$\epsilon_r = 1 + \frac{\omega_p^2}{\omega_0^2 - \omega^2 + i \Gamma \omega}$$
	
	\item Exprimer l'indice complexe $\underline{n}$ dans le cas $\epsilon_r-1 \ll 1$. On définira sa partie réelle $n'$ (indice dispersif) et sa partie imaginaire $n''$ (indice d'absorption). Tracer $n'$ et $n''$ en fonction de $\omega$.
	\item En écrivant le champ électrique pour une onde électromagnétique plane et harmonique, montrer qu'il y a absorption et dispersion et justifier les noms des parties réelle et imaginaire de l'indice.
\end{enumerate}

\subsection{Exercice 3 : Équation de Klein-Gordon et masse du photon}

\begin{enumerate}
	\item Rappeler l'équation de propagation des ondes électromagnétiques dans le vide.
	
	Dans le cadre de la théorie électromagnétique étendue au cas d'un photon de masse non nulle, l'équation de propagation du champ électrique devient $$\Delta \vec{E} - \frac{1}{c^2} \frac{\partial^2 \vec{E}}{\partial t^2} = \eta^2 \vec{E}$$

	\item Quelle est l'unité de $\eta$ ?
	\item Déterminer la relation de dispersion.
	\item Exprimer les vitesses de phase et de groupe en fonction de $c$, $\omega$ et $\eta$. Les tracer. Commentaires ?
	\item Rappeler les expressions de l'énergie $E$ et de la quantité de mouvement $\vec{p}$ d'un photon.
	\item Sachant que pour une particule relativiste de masse $m$, on a $E^2 = p^2 c^2 + m^2 c^4$, exprimer la masse du photon en fonction de $\eta$, $\hbar$ et $c$.
	
	Deux photons de longueurs d'ondes $\lambda_1$ et $\lambda_2$ sont émis au même instant par une source ponctuelle située à une distance $L$. On supposera $\eta^2 \lambda_{1, 2}^2 \ll 1$.
	
	\item Exprimer la différence $\delta t$ des temps de réception des deux signaux.
	
	\item L'observation de certaines étoiles doubles donne $\delta t < \SI{1e-3}{s}$ pour $\lambda_1 = \SI{0.4}{\micro\meter}$, $\lambda_2 = \SI{0.8}{\micro\meter}$ et $L = \SI{1e3}{\textrm{années lumières}}$. En déduire une limite supérieure pour la masse du photon. Commentaire.
	
\end{enumerate}


