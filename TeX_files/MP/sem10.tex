\section{Semaine 10 (02/12-06/12) }

\e{Notions abordées :}
\begin{itemize}
	\item Équations de Maxwell.
	\item Application du premier principe à la transformation chimique.
\end{itemize}

\subsection{Questions de cours}

\begin{enumerate}
	\item Énoncer les équations locales de Maxwell. Démontrer l'équivalent global pour l'équation de Maxwell-Gauss et l'équation de Maxwell-Ampère.
	\item Déterminer l'équation de propagation du champ électromagnétique dans un milieu vide de charges et de courants.
	\item Démontrer, par un bilan de charge électrique, l'équation locale de conservation de la charge.
\end{enumerate}

\subsection{Exercice 1 : Combustion de l'éthyne}

On considère la réaction de combustion de l'éthyne (\ce{C2H2_{(g)}}). On donne $\Delta_RH^\circ(\SI{298}{K}) = \SI{-402}{\kilo\joule\per\mole}$ ainsi que les capacités thermiques à pression constante :

\vspace{5mm}

\begin{tabular}{|c|c|c|c|}
	\hline
	& \ce{CO2_{(g)}} & \ce{H2O_{(g)}} & \ce{H2O_{(l)}} \\ \hline
	$C_p~(\SI{}{\joule\per\kelvin\per\mole})$ & 37.1 & 33.6 & 75.5 \\ \hline
\end{tabular}

\vspace{5mm}

On donne également l'enthalpie de vaporisation de l'eau $\Delta_{vap} H^\circ = \SI{40.7}{\kilo\joule\per\mole}$.

Les réactifs sont introduits dans les proportions stoechiométriques à $T_i = \SI{298}{K}$ et $P=P^\circ$ maintenue constante. Déterminer la température finale $T_f$.

\e{Réponse :} $T_f = \SI{3620}{K}$

\subsection{Exercice 2 : Mesure de l'enthalpie d'autoprotolyse de l'eau}

Une solution contenant $n_0 = \SI{4.55e-4}{\mole}$ d'ions hydronium est ajoutée à une solution de soude concentrée placée dans un calorimètre. La réaction inverse de l'autoprotolyse de l'eau se produit et consomme tous les ions hydronium introduits. Une élévation de température de $\Delta T_a = \SI{0.1028}{\degree C}$ est mesurée. Ensuite, un courant $I=\SI{0.1}{A}$ passant pendant $\delta t = \SI{10.75}{s}$ à travers une résistance $R = \SI{252.7}{\ohm}$ complètement immergée cause une augmentation de température $\Delta T_b = \SI{0.1087}{\degree C}$.

Déterminer l'enthalpie standard de la réaction d'autoprotolyse de l'eau.

\e{Réponse :} $\SI{51.9}{\kilo\joule\per\mole}$

\subsection{Exercice 3 : Combustion du monoxyde de carbone}

Calculer la température maximale de la combustion totale isobare du monoxyde de carbone 
\begin{enumerate}
	\item Avec de l'oxygène en proportions stoechiométriques.
	\item Avec de l'air ($1$ volume de dioxygène pour $3.8$ volumes de diazote, l'oxygène étant toujours en proportions stoechiométriques).
\end{enumerate}

\e{Données :}

\begin{tabular}{|c|c|c|c|c|}
	\hline
	& \ce{CO2} & \ce{CO} & \ce{O2} & \ce{N2} \\ \hline
	$\Delta_fH^\circ(\SI{298}{K})~(\SI{}{\kilo\joule\per\mole})$ & $-395.5$ & $-110.4$ & & \\ \hline
	$c_p^\circ~(\SI{}{\joule\per\kelvin\per\mole})$ & 30.5 & 26.9 & 27.2 & 27.2 \\ \hline
\end{tabular}