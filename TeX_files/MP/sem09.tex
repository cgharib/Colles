\section{Semaine 09 (25/11-29/11) }

\e{Notions abordées :}
\begin{itemize}
	\item Induction (révisions de MPSI) (priorité).
	\item Magnétostatique.
	\item Dipôles électro- et magnétostatiques.
\end{itemize}

\subsection{Exercice 1 : Double rails de Laplace}

Deux barres parallèles conductrices, de résistance $R$, sont posées à l'horizontale sur deux rails parallèles et conducteurs, séparés d'une distance $l$. L'une des barres est animée d'une vitesse $\vec{V}$. L'ensemble baigne dans un champ magnétique vertical homogène et stationnaire $\vec{B}$. 

\begin{enumerate}
	\item On suppose la vitesse $\vec{V}$ constante. Déterminer le mouvement de l'autre barre, d'abord qualitativement, puis quantitativement.
	\item Étudier le cas du régime sinusoïdal forcé.
\end{enumerate}

\subsection{Exercice 2 : Inductances propres et mutuelles}

\begin{enumerate}
	\item Rappeler la définitions des inductances propres et mutuelles. À quoi ces grandeurs nous servent-elles ?
	\item Calculer l'inductance propre d'un solénoïde de longueur $l$, de section $S$, comportant $N$ spires, et supposé suffisamment long pour négliger les effets de bords. Application numérique pour une bobine de TP.
	\item Exprimer l'énergie magnétique stockée dans la bobine. Retrouver l'expression de la densité volumique d'énergie magnétique.
	\item On considère un petit solénoïde à l'intérieur d'un gros solénoïde, les deux partageant le même axe. Déterminer l'inductance mutuelle.
\end{enumerate}

\e{Réponses :}
\begin{enumerate}
	\item -
	\item $L = \mu_0 \frac{N^2 S}{l}$
	\item -
	\item $M = \mu_0 \frac{N_1 N_2}{l_1}S_1$
\end{enumerate}

\subsection{Exercice 3 : Induction par un aimant mobile}

Une spire circulaire d'axe $(Oz)$, de rayon $a$ et de résistance $R$ est immobile. Sur son axe, on rapproche à vitesse $\vec{V} = V \vec{e_z}$ constante un petit aimant assimilé à un dipôle magnétique de moment $\vec{\mathcal{M}} = \mathcal{M}\vec{e_z}$.

\begin{enumerate}
	\item Prévoir le sens du courant $i$ induit dans la spire.
	\item Calculer le courant $i$ dans la spire.
	\item Quelle est la force exercée par l'aimant sur la spire ? Commenter.
\end{enumerate}

\subsection{Exercice 4 : Mesure d'une inductance mutuelle par battements}

On considère deux circuits $LC$ identiques et couplés par une inductance mutuelle $M$. On suppose qu'initialement le condensateur $1$ porte la charge $Q$, que le condensateur $2$ est déchargé et qu'aucun courant ne circule.

\begin{enumerate}
	\item Déterminer les équations sur les charges portées par les condensateurs.
	\item Découpler le système d'équations et résoudre chacune des deux équations différentielles.
	\item Déterminer le courant dans le circuit $1$ en fonction du temps. En déduire une méthode pour mesurer $M$.
\end{enumerate}