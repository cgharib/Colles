\section{Semaine 17 (03/02-07/02) }

\e{Notions abordées :}
\begin{itemize}
	\item Réflexion sur un conducteur parfait (cf. semaine précédente).
	\item Rayonnement dipolaire électrique.
	\item Thermodynamique de MPSI.
	\item Systèmes thermodynamiques en écoulement.
\end{itemize}


\subsection{Exercice 1 : Modèle planétaire de l'atome}

On considère un électron en mouvement circulaire uniforme de rayon $r_0 = \SI{1e-10}{m}$ à la vitesse angulaire $\omega$ autour d'un proton immobile. Soit $\omega$ la pulsation de ce mouvement de rotation.

\begin{enumerate}
	\item Exprimer $\omega$ en fonction de $e$, $\epsilon_0$, $m_e$ et $r_0$.
	\item Déterminer le dipôle électrique formé par ce système. Montrer qu'il s'agit d'une superposition de deux dipôles oscillants perpendiculaires. Déterminer leur amplitude $P_0$. Application numérique.
	\item Exprimer l'énergie mécanique du système en fonction de $r$, $\epsilon_0$ et $r$.
	\item Rappeler, ou retrouver en partie, les expressions du champ électromagnétique rayonné par un dipôle oscillant.
	\item En déduire l'expression du vecteur de Poynting. Calculer sa valeur moyenne en fonction de $P_0$, $\mu_0$, $\theta$, $\omega$, $r$ et $c$.
	\item En déduire l'énergie rayonnée par unit é de temps par l'électron.
	\item Justifier qualitativement que l'électron doit s'écraser sur le noyau. 
	\item Soit $r(t)$ le rayon de la trajectoire de l'électron Par un bilan d'énergie, déterminer l'équation différentielle sur $r(t)$.  
	\item En déduire la durée de vie de l'atome d'hydrogène dans ce modèle. Application numérique. Commentaire.
\end{enumerate}

\subsection{Exercice 2 : Propagation guidée entre deux plans}

Deux plans infinis conducteurs parfaits délimitent une cavité vide entre $z=0$ et $z=b$. 

\begin{enumerate}
	\item Rappeler l'équation de propagation du champ électromagnétique dans la cavité.
	\item Justifier que, contrairement à ce qu'on fait d'habitude, on ne peut pas chercher une solution sous la forme d'une OPPH.
	
	On cherche une solution de l'équation sous la forme $$\vec{E}(x, z, t) = \beta(z)\vec{u_y}\cos(\omega t - k x)$$.
	
	\item Déterminer l'équation différentielle vérifiée par $\beta$. En donner la solution générale.
	\item Montrer que l'onde ne peut exister que si $\omega > kc$. Commenter en relation avec la relation de dispersion habituelle.
	\item Déterminer complètement $\beta(z)$.
	\item Déterminer la relation de dispersion. 
	\item En déduire qu'il existe une pulsation minimale pour qu'une onde électromagnétique se propage dans le guide.
\end{enumerate}

\e{Réponse :} $\vec{k}^2 = \frac{\omega^2}{c^2} - \frac{n^2 \pi^2}{b^2}$

\subsection{Exercice 3 : Lois de Descartes sur la réflexion et la réfraction}

L'objectif de cet exercice est de redémontrer les lois de Descartes à l'interface entre deux diélectriques.

On rappelle que dans un diélectrique, tout se passe comme dans le vide, à condition de remplacer $\epsilon_0$ par $\epsilon = \epsilon_0 \epsilon_r$.

On considère donc deux diélectriques accolés. Le milieu $1$ de permittivité $epsilon_1$ occupe le demi-espace $z<0$. Le milieu $2$ de permittivité $epsilon_2$ occupe le demi-espace $z>0$.

On considère également une OPPH électromagnétique incidente $(\vec{E_i}, \vec{B_i}, \omega_i, \vec{k_i} = k_{i,x}\vec{e_x} + k_{i, z}\vec{e_z})$ provenant du milieu $1$ vers l'interface avec un angle $\alpha$ par rapport à la normale. Pour simplifier, on suppose le champ électrique incident polarisé rectilignement selon la perpendiculaire au plan d'incidence.

On rappelle qu'à une interface, la composante tangentielle du champ électrique est continue.

\begin{enumerate}
	\item Rappeler les trois lois de Descartes.
	
	\item Justifier qu'il doit exister une onde transmise et/ou une onde réflechie. 

	On appellera $(\vec{E_t}, \vec{B_t}, \omega_t, \vec{k_t})$ l'onde transmise et $(\vec{E_r}, \vec{B_r}, \omega_r, \vec{k_r})$ l'onde réfléchie. 
	
	\item Justifier qualitativement que, si ils existent, les champ électrique transmis et réfléchi ont la même polarisation que le champ électrique incident.
	
	\item Sur un schéma, faire figurer l'interface, les trois ondes ainsi que les angles respectifs.
	
	\item Montrer que les ondes ont toute la même pulsation.
	
	\item Montrer que les ondes ont toute le même $k_x$.
	
	\item Montrer que $\frac{\vec{k_t}^2}{\epsilon_{r2}} = \frac{\vec{k_r}^2}{\epsilon_{r1}} = \frac{\vec{k_i}^2}{\epsilon_{r1}}$.
	
	\item En déduire les lois de Descartes sur la réflexion et la réfraction.
\end{enumerate}