\section{Semaine 14 (13/01-17/01) }

\e{Notions abordées :}
\begin{itemize}
	\item Thermochimie.
	\item Ondes électromagnétiques dans le vide.
\end{itemize}

\subsection{Questions de cours}

\begin{enumerate}
	\item Établir l'équation de propagation des ondes électromagnétiques dans le vide.
	\item Établir l'équation de dispersion pour une OPPH électromagnétique dans le vide.
	\item Démontrer la relation de structure pour des OPPH électromagnétiques.
\end{enumerate}

\subsection{Exercice 1 : Déplacement d'équilibre par ajout d'un composé inerte}

On considère la réaction de dismutation $$\ce{2NaHCO3(s) = Na2CO3(s) + CO2(g) + H2O(g)}.$$

La réaction a lieu dans une enceinte de volume $V = \SI{50}{L}$ qui contient initialement $n = \SI{2.0}{mol}$ de \ce{NaHCO3(s)} (la pression initiale est donc nulle).

\begin{enumerate}
	\item À $\SI{47}{\celsius}$, la pression d'équilibre vaut $\SI{0.033}{bar}$. Calculer $K^\circ(\SI{47}{\celsius})$.
	\item À $\SI{77}{\celsius}$, la pression d'équilibre vaut $\SI{0.265}{bar}$. Calculer $\Delta_rH^\circ$ et $\Delta_rS^\circ$.
	\item Donner l'état final à $\SI{107}{\celsius}$.
	\item L'équilibre étant atteint, on ajoute du diazote dans l'enceinte. Que se passe-t-il si l'ajout se fait à volume constant ? À pression constante ?
\end{enumerate}

\e{Réponses :}
\begin{enumerate}
	\item $K^\circ = \SI{2.722e-4}{}$
	\item $\Delta_rH^\circ = \SI{129.3}{\kilo\joule\per\mole}$ et $\Delta_rS^\circ = \SI{335.9}{\joule\per\kelvin\per\mole}$
	\item $\xi_f = \SI{1.22}{mol}$
	\item -
\end{enumerate}

\subsection{Exercice 2 : Déchloration du pentachlorure de phosphore}

On étudie la réaction $$\ce{PCl5(g) = PCl3(g) + Cl2(g)}.$$

\begin{enumerate}
	\item Calculer $K^\circ(\SI{500}{K})$.
	\item Sous $P = \SI{3.0}{bar}$, on mélange $\SI{0.15}{mol}$ de \ce{PCl5(g)}, $\SI{0.40}{mol}$ de \ce{PCl3(g)} et $\SI{0.10}{mol}$ de \ce{Cl2(g)}.
	\begin{enumerate}
		\item Dans quel sens évolue le système ? 
		\item Déterminer la composition à l'équilibre.
	\end{enumerate}
	\item À partir d'un équilibre, comment évolue le système si :
	\begin{enumerate}
		\item on augmente $T$ à $P$ constante ?
		\item on augmente $P$ à $T$ constante ?
		\item on augmente $T$ à $V$ constant ?
		\item on introduit du dichlore à $T$ et $P$ constantes ?
	\end{enumerate}
\end{enumerate}

\e{Données :}

\begin{tabular}{|c|c|c|c|}
	\hline
	& \ce{PCl5_{(g)}} & \ce{PCl3_{(g)}} & \ce{Cl2_{(g)}}\\ \hline
	$\Delta_fH^\circ(\SI{298}{K})~(\SI{}{\kilo\joule\per\mole})$ & $-374.9$ & $-287.0$ & \\ \hline
	$S^\circ_{mol}~(\SI{}{\joule\per\kelvin\per\mole})$ & 364.5 & 311 & 223 \\ \hline
\end{tabular}

\e{Réponses :}

\begin{enumerate}
	\item $K^\circ(\SI{500}{K}) = \SI{0.4687}{}$.
	\item -
	\begin{enumerate}
		\item $Q_r = 1.231$
		\item $\xi_f = \SI{-0.0473}{mol}$
	\end{enumerate}
	\item -
	\begin{enumerate}
		\item Endothermique.
		\item Rétrograde.
		\item Direct.
		\item Rétrograde.
	\end{enumerate}
\end{enumerate}

\subsection{Exercice 3 : Combustion du méthane}

On considère la combustion d'un volume $V_0 = \SI{1.00}{m^3}$ de méthane à la pression $P^\circ$ et à la température $T_0 = \SI{298}{K}$.

\begin{enumerate}
	\item Écrire la réaction de combustion du méthane dans le dioxygène.
	\item Montrer que l'on peut considérer la réaction comme totale.
	\item Calculer la quantité de matière $n_0$ de méthane contenue dans l'enceinte.
	\item Calculer l'énergie libérée par la combustion isotherme isobare de cette quantité de méthane.
	\item On ne suppose plus la réaction totale. Comment évolue l'équilibre si on augmente, toutes choses égales par ailleurs, la pression. 
	\item Comment évolue l'équilibre si on augmente la pression, dans le cas de la combustion de l'éthane ?
\end{enumerate}

\e{Données :}

\begin{tabular}{|c|c|c|c|c|}
	\hline
	& \ce{CH4_{(g)}} & \ce{O2_{(g)}} & \ce{CO2_{(g)}} & \ce{H2O_{(l)}} \\ \hline
	$\Delta_fH^\circ(\SI{298}{K})~(\SI{}{\kilo\joule\per\mole})$ & -74.4 & & -393.5 & -285.8 \\ \hline
	$S^\circ_{mol}~(\SI{}{\joule\per\kelvin\per\mole})$ & 186.2 & 205.0 & 213.6 & 69.9 \\ \hline
\end{tabular}

\e{Réponses :}
\begin{enumerate}
	\item -
	\item $K^\circ = $
	\item $n_0 = \SI{40.4}{mol}$
	\item $Q = \SI{3.6e4}{\kilo\joule}$
	\item Aucun effet.
	\item Rétrograde.
\end{enumerate}