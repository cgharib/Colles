\section{Semaine 15 (20/01-24/01) }


\e{Notions abordées :}
\begin{itemize}
	\item Travail et énergie (cf. semaine précédente).
	\item Mouvements de particules dans un champ électrique.
\end{itemize}

\subsection{Questions de cours}

\begin{enumerate}
	\item Donner l'expression de la force électrostatique de Coulomb et l'énergie potentielle associée.
	\item Montrer qu'un champ magnétique seul ne peut pas modifier l'énergie cinétique d'une particule chargée.
	\item Exprimer la variation d'énergie potentielle associée à la force électrique (en fonction de la variation du potentiel électrique).
	
\end{enumerate}

\subsection{Exercice 1 : Accélération et déviation d'un électron par différence de potentiel}

Soient deux plaques planes horizontales écartées d'une distance $d$ et chargées de charges opposées. La plaque positive est au-dessus.

\begin{enumerate}
	\item Déterminer, en explicitant les approximations nécessaires, les caractéristiques du champ électrique pour $U = \SI{1000}{V}$ et $d = \SI{10}{cm}$.
	\item Un électron de masse $m = \SI{9.1e-31}{kg}$ et de charge $-e = -\SI{1.6e-19}{C}$ pénètre à une distance $d' = \SI{2.0}{cm}$ de la plaque négative avec une vitesse d'entrée $\vec{v_0}$ parallèle aux plaques et de norme $v_0 = \SI{2.0e4}{\meter\per\second}$. Justifier que le poids est négligeable.
	\item Déterminer la longueur $l$ que doivent avoir les plaques pour que l'électron atteigne la plaque positive.
	\item Expliciter les caractéristiques (norme et inclinaison) de la vitesse de l'électron quand il arrive sur la plaque positive.
\end{enumerate}

\subsection{Exercice 2 : Accélération de particules $\alpha$}

Les particules $\alpha$ sont des noyaux d'hélium, de masse $m = \SI{6.6e-27}{kg}$ qui sont émises par radioactivité et qui sont beaucoup utilisées en physique des particules. On considère un faisceau de particules $\alpha$ de vitesse $v_0 = \SI{2000}{\meter\per\second}$ qui pénètrent dans une zone où règne un champ électrostatique uniforme $\vec{E}$ d'intensité $\SI{1000}{V\per\meter}$. Le champ électrostatique et la vitesse initiale sont colinéaires de sens opposé. 

On rappelle la valeur de la charge élémentaire  $e = \SI{1.6e-19}{C}$

\begin{enumerate}
	\item Quelle est la charge d'une particule $\alpha$ ?
	\item Montrer que l'on peut négliger le poids de la particule.
	\item Décrire le mouvement d'une particule $\alpha$. Déterminer le point de demi-tour.
	\item Expliciter la durée passée par la particule $\alpha$ dans la zone où règne le champ électrostatique ainsi que les caractéristiques de sa vitesse quand elle ressort.
	\item Par une analyse énergétique, retrouver la distance parcourue avant le demi-tour ainsi que la vitesse à la sortie du champ.
\end{enumerate}

\subsection{Exercice 3 : Principe d'un oscilloscope analogique}

Un oscilloscope analogique est constitué d'un canon à électrons et d'une zone de déviation. On rappelle la masse de l'électron $m = \SI{9.1e-31}{kg}$ et sa charge $-e = -\SI{1.6e-19}{C}$. 

\begin{enumerate}
	\item Considérons tout d'abord le canon à électrons. Il permet d'accélérer des électrons d'une vitesse négligeable à une vitesse $v_0$, en leur appliquant une tension $V = \SI{600}{V}$. Déterminer, en justifiant les approximations, la vitesse $v_0$. La mécanique classique est-elle encore valable ?
	
	\item À la sortie du canon, les électrons pénètrent à la vitesse $v_0$ entre deux plaques planes horizontales entre lesquelles on applique la tension $U=\SI{2.0}{V}$ prélevée en entrée de l'oscilloscope. On suppose que le faisceau d'électrons arrive à égale distance des deux plaques avec une vitesse horizontale. Les plaques sont de longueur $l = \SI{25}{mm}$ et distantes de $d = \SI{10}{mm}$. 
	\begin{enumerate}
		\item Sur un schéma, faire figurer les plaques et leurs dimensions, leur potentiel, la tension $U$, le champ électrique $\vec{E}$ et le faisceau d'électrons.
		\item Déterminer les équations du mouvement d'un électron.
		\item En déduire l'ordonnée $y_S$ où se produit la sortie des plaques.
		\item Déterminer la vitesse de l'électron à la sortie des plaques.
	\end{enumerate}
	
	\item Finalement, on place un écran à une distance $D = \SI{10}{cm}$ de l'extrémité des plaques. Quelle est la position $y_E$ du point d'impact de l'électron sur l'écran ? Commenter l'expression en regard de l'utilité d'un oscilloscope. Commenter la valeur numérique.

\end{enumerate}

\e{Réponse :} $y_E = \frac{e U l}{m d v_0^2}\left(D + \frac{l}{2}\right) = \SI{0.50}{mm}$