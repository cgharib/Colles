\section{Semaine 28 (19/05 - 23/05) }


\e{Notions abordées :}
\begin{itemize}
	\item Introduction à la physique quantique. (cf semaine précédente).
	\item Introduction à la thermodynamique.
	\item Corps pur diphasé.
\end{itemize}

\subsection{Questions de cours}

\begin{enumerate}
	\item Définir l'énergie interne et l'enthalpie.
	\item Définir les capacités thermiques. Valeurs pour des gaz parfaits mono- et diatomiques.
	\item Lois de Joule. Démonstration (à partir de la température cinétique).
\end{enumerate}

\subsection{Exercice 1 : Gonfler un pneu}

On considère un pneu de vélo de volume $V = \SI{2.5}{\liter}$ supposé constant. Le pneu est initialement dégonflé et l'air qui s'y trouve est à la pression atmosphérique à température ambiante. On souhaite gonfler ce pneu jusqu'à une pression $P_f = \SI{6.0}{bar}$ avec une pompe manuelle pouvant contenir un volume $V_a = \SI{300}{\cubic\centi\meter}$ d'air au maximum.

Lorsqu'on tire le piston de la pompe vers le haut, on admet un volume $V_a$ d'air à la pression ambiante. Puis, lors de l'étape de refoulement, l'air admis est intégralement transvasé dans le pneu de volume $V$ invariable. La pression augmente donc.

\begin{enumerate}
	\item Justifier que si l'on procède lentement la température $T_0$ de l'air dans le pneu n'évolue pas.
	\item Déterminer la pression dans le pneu après le premier coup de pompe en fonction de $P_0$, $V_a$ et $V$.
	\item Calculer la pression $P_n$ dans le pneu après $n$ coups de pompes en fonction de $n$, $P_0$, $V_a$ et $V$.
	\item Combien de coups de pompes sont nécessaires pour gonfler le pneu à la pression désirée ?
\end{enumerate}

\e{Réponses :}
\begin{enumerate}
	\item -
	\item $P_1 = P_0\left( 1 + \frac{V_a}{V} \right)$
	\item $P_n = P_0\left( 1 + n\frac{V_a}{V} \right)$
	\item $42$.
\end{enumerate}

\subsection{Exercice 2 : Diagramme $(P, T)$ de l'eau}

Soit une masse $m = \SI{900}{\gram}$ d'eau subissant depuis un état initial à la température $T_1 = \SI{298}{\kelvin}$ et à la pression $P_1 = \SI{1.00}{bar}$ un chauffage à pression constante jusqu'à la température $T_2 = \SI{473}{\kelvin}$ puis une compression à température constante jusqu'à un volume $V_3 = \SI{100}{\liter}$ et enfin une autre compression à température constante jusqu'à un état d'équilibre liquide-vapeur de volume $V_4 = \SI{50.0}{\liter}$. On donne les pressions de vapeur saturante à $\SI{373}{\kelvin}$ et $\SI{473}{\kelvin}$ qui sont respectivement $\SI{1.00}{bar}$ et $\SI{15.5}{bar}$ ainsi que la masse molaire de l'eau $\SI{18}{\gram\per\mol}$.

\begin{enumerate}
	\item Quel est l'état physique de l'eau dans les états 1 et 2 ?
	\item Déterminer la pression $P_3$ de l'état 3 ainsi que l'état physique de l'eau dans cet état.
	\item Déterminer le titre en vapeur de l'eau dans l'état 4.
	\item Représenter les différentes transformations dans le diagramme donnant la pression en fonction de la température.
\end{enumerate}

\e{Réponses :}
\begin{enumerate}
	\item liquide pour 1, vapeur pour 2
	\item $P_3 = P_{sat}(T_3)$ et $x_{V, 3} = 0.788$
	\item $x_{V, 4} = 0.394$
\end{enumerate}

\subsection{Exercice 3 : États de l'éther}

On conserve dans une pièce à $\SI{18.0}{\degreeCelsius}$ un flacon contenant $\SI{50.0}{\milli\liter}$ d'éther liquide à cette température et à la pression de vapeur saturante $P_{sat} = \SI{0.544}{bar}$. On suppose que le flacon ne contient que de l'éther.

On donne les caractéristiques physiques suivantes pour l'éther : 

\begin{tabular}{|c|c|c|}
	\hline
	$T~(\unit{\degreeCelsius})$ & $P_{sat}$ (bar) & $\rho_{liquide}~(\unit{\kilogram\per\liter})$ \\ \hline
	$18.0$ & $0.544$ & $0.716$ \\ \hline
	$49.0$ & $1.65$ & $0.679$ \\ \hline
\end{tabular}

La pression du point critique est $\SI{36.4}{bar}$ et sa température $\SI{194}{\degreeCelsius}$. La masse molaire de l'éther est $M = \SI{74.1}{\gram\per\mole}$.

\begin{enumerate}
	\item Dessiner l'allure du diagramme de Clapeyron pour l'équilibre liquide-vapeur de l'éther. Faire figurer une isotherme ainsi que les points d'intérêt. Nommer les différentes courbes qui apparaissent.
	\item Déterminer la masse et la quantité de matière d'éther contenu dans ce flacon à $\SI{18.0}{\degreeCelsius}$.
	\item Déterminer les volumes massiques du liquide saturant et de la vapeur saturante à cette température. 
	\item Quels sont les volumes du flacon permettant d'avoir un mélange liquide-vapeur d'éther ?
	\item Pour un volume $V = \SI{5.50}{\liter}$, déterminer le volume massique. En déduire la fraction massique de la vapeur. La fraction molaire a-t-elle une valeur différente ?
	\item Quel est l'état du système si on augmente la température jusqu'à $\SI{49}{\degreeCelsius}$ ? On déterminera la fraction de vapeur si on a un mélange liquide-vapeur, ou la pression si on a un système gazeux.
	\item Même questions pour un volume $V' = \SI{10.0}{\liter}$ à $\SI{18}{\degreeCelsius}$ puis $\SI{49}{\degreeCelsius}$.
\end{enumerate}

\e{Réponses :}
\begin{enumerate}
	\item -
	\item $m = \SI{35.8}{\gram}$ et $n = \SI{0.483}{\mole}$.
	\item $v_l = \SI{1.40}{\liter\per\kilogram}$ et $v_v = \SI{600}{\liter\per\kilogram}$.
	\item $\SI{50.0}{\milli\liter} < V < \SI{21.5}{\liter}$
	\item $v = \SI{154}{\liter\per\kilogram}$, $x_v = \SI{25.5}{\percent}$
	\item $x_v = \SI{70.1}{\percent}$
	\item D'abord $x_v = \SI{46.5}{\percent}$, ensuite $P = \SI{1.29}{bar}$.
\end{enumerate}