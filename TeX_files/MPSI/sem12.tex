\section{Semaine 12 (16/12-20/12) }


\e{Notions abordées :}
\begin{itemize}
	\item Structure des entités chimiques (cf semaine précédente).
	\item Cinématique du point matériel.
\end{itemize}

\subsection{Questions de cours}

\begin{enumerate}
	\item Expression du vecteur position $\vec{OM}$ en coordonnées cartésiennes, polaires et cylindrique + dessin.
	\item Expression du vecteur déplacement infinitésimal $\vec{dOM}$ en coordonnées cartésiennes, polaires et cylindriques.
	\item Expression de l'accélération en coordonnées cartésiennes, polaires et cylindriques.
\end{enumerate}

\subsection{Exercice 1 : Fronde}

Un jeune garçon s’amuse à faire tourner un caillou accroché au bout d’une corde de longueur $R = \SI{1.2}{m}$ dans un plan horizontal à une hauteur $h=\SI{1.8}{m}$ au-dessus du sol selon un mouvement circulaire. La vitesse devenant trop grande, la corde se casse et le caillou part horizontalement pour tomber à une distance $d = \SI{9.1}{m}$ de son point de décrochage. Il est soumis durant cette phase à la seule accélération de pesanteur $g$ qu’on prendra égale en norme à $g = \SI{9.8}{\meter\per\second\squared}$. 

Quelle était l’accélération centripète radiale au moment de la rupture ? Commentaire.

\e{Réponse :} $\SI{190}{\meter\per\second\squared}$

\subsection{Exercice 2 : Balle lancée depuis une montgolfière}

Une montgolfière se déplace à l'altitude $h = \SI{100}{m}$ constante avec une vitesse horizon-
tale de 10 m.s−1 . Un passager lance une balle vers le haut suivant la verticale mais du
fait de l’entraînement horizontal dû au déplacement de la montgolfière, la vitesse ini-
tiale résultante de la balle est inclinée d’un angle α = 40◦ par rapport à l’horizontale
pour quelqu’un observant la chute depuis le sol. La balle subit alors une accélération
descendante constante g = 9,8 m.s−2 .
a) Quelle est la durée de la chute ?
b) Déterminer le lieu où la balle touche le sol.
c) Déterminer la vitesse avec laquelle la balle arrive au sol.
d) Comparer les déplacements horizontaux de la balle et de la montgolfière. En dé-
duire la nature de la trajectoire de la balle pour le passager de la montgolfière.

\subsection{Exercice 3 : Cycliste sur piste}

La piste d’un vélodrome est constituée de deux demi-cercles de rayon R = 23 m rac-
cordés par deux portions rectilignes de longueur ℓ = 53 m.
a) Calculer la longueur de la piste.
b) Un cycliste sur piste est-il en rotation ou en translation circulaire ?
c) Il atteint une vitesse de 51 km.h−1 sur la partie circulaire de rayon R = 10 m. Dé-
terminer sa vitesse angulaire.
d) Déterminer la fréquence du mouvement sachant que la longueur de la partie rec-
tiligne est ℓ = 68 m.
