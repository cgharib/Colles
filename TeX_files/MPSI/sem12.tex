\section{Semaine 12 (16/12-20/12) }


\e{Notions abordées :}
\begin{itemize}
	\item Structure des entités chimiques (cf semaine précédente).
	\item Cinématique du point matériel.
\end{itemize}

\subsection{Questions de cours}

\begin{enumerate}
	\item Expression du vecteur position $\vec{OM}$ en coordonnées cartésiennes, polaires et cylindrique + dessin.
	\item Expression du vecteur déplacement infinitésimal $\vec{dOM}$ en coordonnées cartésiennes, polaires et cylindriques.
	\item Expression de l'accélération en coordonnées cartésiennes, polaires et cylindriques.
\end{enumerate}

\subsection{Exercice 1 : Fronde}

Un jeune garçon s’amuse à faire tourner un caillou accroché au bout d’une corde de longueur $R = \SI{1.2}{m}$ dans un plan horizontal à une hauteur $h=\SI{1.8}{m}$ au-dessus du sol selon un mouvement circulaire. La vitesse devenant trop grande, la corde se casse et le caillou part horizontalement pour tomber à une distance $d = \SI{9.1}{m}$ de son point de décrochage. Il est soumis durant cette phase à la seule accélération de pesanteur $g$ qu’on prendra égale en norme à $g = \SI{9.8}{\meter\per\second\squared}$. 

Quelle était l’accélération centripète radiale au moment de la rupture ? Commentaire.

\e{Réponse :} $\SI{190}{\meter\per\second\squared}$

\subsection{Exercice 2 : Balle lancée depuis une montgolfière}

Une montgolfière se déplace à l'altitude $h = \SI{100}{m}$ constante avec une vitesse horizontale de $\SI{10}{\meter\per\second}$. Un passager lance une balle vers le haut suivant la verticale mais du fait de l’entraînement horizontal dû au déplacement de la montgolfière, la vitesse initiale résultante de la balle est inclinée d’un angle $\alpha = \SI{40}{\degree}$ par rapport à l’horizontale pour quelqu’un observant la chute depuis le sol. La balle subit alors une accélération descendante constante $g = \SI{9.8}{\meter\per\second\squared}$.

\begin{enumerate}
	\item Quelle est la durée de la chute ?
	\item Déterminer le lieu où la balle touche le sol.
	\item Déterminer la vitesse avec laquelle la balle arrive au sol.
	\item Comparer les déplacements horizontaux de la balle et de la montgolfière. En déduire la nature de la trajectoire de la balle pour le passager de la montgolfière.
\end{enumerate}

\subsection{Exercice 3 : Cycliste sur piste}

On s'intéresse à un cycliste, considéré comme un point matériel $M$ qui s'entraîne sur un vélodrome constitué de deux demi-cercles reliés par deux lignes droites. La longueur des parties rectiligne est $L = \SI{62}{m}$. Le rayon des demi-cercles est $R = \SI{20}{m}$. Le cycliste part du point $D$, au milieu d'une des parties rectilignes, avec une vitesse nulle.

\begin{enumerate}
	\item Il exerce un premier effort, ce qui se traduit par une accélération constante $a_1$ jusqu'à l'entrée $E_1$ du premier virage. Calculer le temps $t_{E1}$ de passage en $E_1$ ainsi que la vitesse $V_{E1}$ en fonction de $a_1$ et $L$.
	\item Dans le premier virage, le cycliste a une accélération tangentielle constante et égale à $a_1$. Déterminer le temps $t_{S1}$ de passage en $S_1$ (sortie du virage) ainsi que la vitesse $v_{S1}$ en fonction de $a_1$, $L$ et $R$.
	\item De même, en considérant l'accélération tangentielle constante tout au long du premier tour et égale à $a_1$, déterminer les temps $t_{E2}$, $t_{S2}$ et $t_D$ (après un tour), ainsi que les vitesses correspondantes.
	\item La course s'effectue sur quatre tours ($\SI{1}{km}$) mais on ne s'intéresse qu'au premier effectué en $t_1 = \SI{18.155}{s}$ (Chris Hoy aux championnats du monde de 2007). Déterminer la valeur de l'accélération $a_1$ ainsi que la vitesse atteinte en $D$. La vitesse mesurée sur piste est d'environ $\SI{60}{\km\per\hour}$. Que doit on modifier dans le modèle pour se rapprocher de la réalité ?
\end{enumerate}

\subsection{Exercice 4 : Mouvement sur une ellipse}

Un point $M$ se déplace sur une ellipse d'équation cartésienne $\left(\frac{x}{a}\right)^2 + \left(\frac{y}{b}\right)^2 = 1$. On note $\theta$ l'angle que fait $\vec{OM}$ avec l'axe $(Ox)$.

Les coordonnées de $\vec{OM}$ peuvent s'écrire 
$\left\{\begin{array}{ccc}
	x(t) & = & \alpha \cos{\omega t + \phi} \\
	y(t) & = & \beta \sin{\omega t + \psi},
\end{array}\right.$
, où l'on suppose que $\omega$ est une constante.

\begin{enumerate}
	\item À $t=0$, le mobile est en $M_0$, sur l'axe $(Ox)$ et à une distance $a$ de $O$. Déterminer $\alpha$, $\phi$ et $\psi$.
	\item Des autres données, déduire $\beta$.
	\item Déterminer les composantes de la vitesse et de l'accélération.
	\item Montrer que l'accélération est de la forme $\vec{a} = -\omega^2 \vec{OM}$. Commenter.
\end{enumerate}

