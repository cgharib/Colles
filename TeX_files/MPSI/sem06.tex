\section{Semaine 06 (04/11-08/11) }


\e{Notions abordées :}
\begin{itemize}
	\item Systèmes optiques (cf semaine précédente).
	\item Cinétique chimique.
\end{itemize}

\subsection{Questions de cours}

\begin{enumerate}
	\item Définir la vitesse d'une réaction chimique. Quelle est son unité ?
	\item Qu'est ce que l'ordre d'une réaction chimique ? L'ordre partiel ? La constante de vitesse ?
	\item Donner la loi d'Arrhenius.
\end{enumerate}

\subsection{Exercice 1 : Décomposition de l'azométhane en phase gazeuse}

Dans un récipient de volume fixé $V$, on introduit à $\SI{600}{K}$ de l'azométhane \ce{CH3N2CH3_{(g)}}. Celui-ci se décompose en éthane et en diazote gazeux. 

L'évolution de la réaction est suivie par manométrie et une série de mesures a donné la pression partielle $p_A$ en azométhane : 

\begin{tabular}{|c|c|c|c|c|c|}
	\hline 
	$t$ ($10^3$ s) & $0$ & $1.00$ & $2.00$ & $3.00$ & $4.00$ \\ \hline
	$p_A$ ($10^{-2}$ mmHg) & $p_0 = 8.21$ & $5.74$ & $4.00$ & $2.80$ & $1.96$ \\ \hline
\end{tabular}

\begin{enumerate}
	\item Écrire l'équation bilan de la réaction.
	\item Vérifier que la réaction est d'ordre $1$ par rapport au réactif et calculer sa constante de vitesse.
\end{enumerate}

\e{Réponse :} $k = \SI{3.58e-4}{\per\second}$

\subsection{Exercice 2 : Temps de demi-réaction}

La réaction de décomposition totale du pentaoxyde de diazote \ce{N2O5} en dioxyde d'azote \ce{NO2} et dioxygène a lieu en phase gazeuse. L’expérience est menée dans un récipient de volume V constant, initialement vide, en amenant du pentaoxyde de diazote de manière à ce que la pression initiale soit $p_0$.

\begin{enumerate}
	\item On mesure la pression $p(t)$ au cours du temps. On veut évaluer la constante cinétique en mesurant le temps de demi-réaction. Quelle doit être la lecture de $p$ sur le manomètre pour ce temps ?
	\item Le tracé de la courbe $\ln{p(\ce{N2O5})}$ en fonction du temps est une droite. En déduire l'ordre de la réaction. Tracer l'allure de la pression en fonction du temps.
	\item Une première mesure réalisée à $\theta = \SI{150}{\celsius}$ permet de mesurer un temps de demi réaction $t_{1/2} = \SI{7.5}{s}$. Une seconde mesure réalisée à $\theta' = \SI{100}{\celsius}$ permet de mesurer un temps de demi-réaction $t'_{1/2} = \SI{7.0}{min}$. Calculer la constante de vitesse pour ces deux températures.
	\item Calculer l'énergie d'activation de la réaction.
\end{enumerate}

\e{Réponses :}
\begin{enumerate}
	\item $p_{1/2} = \frac{7}{4}p_0$.
	\item -
	\item $k = \SI{9.2e-2}{\per\second}$ et $k' = \SI{1.7e-3}{\per\second}$.
	\item $E_a = \SI{1.1e2}{\kilo\joule\per\mol}$.
\end{enumerate}

\subsection{Exercice 3 : Dismutation des ions hypochlorites}

En solution aqueuse, les ions hypochlorite \ce{ClO-} peuvent se dismuter selon la réaction totale $$\ce{ClO- = \frac{1}{3} ClO3- + \frac{2}{3}Cl-}$$.

La vitesse de la réaction $r$, définie comme la vitesse de disparition des ions hypochlorite \ce{ClO-} suit une loi cinétique de second ordre, dont la constante de vitesse est notée $k$.

On provoque cette réaction dans une solution contenant initialement des ions hypochlorite à la concentration $c_0 = \SI{0.10}{\mol\per\liter}$.

À $T = \SI{343}{K}$, la constante de vitesse de la solution est $k = \SI{3.1e-3}{\per\mol \deci \meter \cubed \per \second}$. 

L’énergie d’activation de cette réaction au voisinage des températures considérées ici est $E_a = \SI{47}{\kilo\joule\per\mol}$.

\begin{enumerate}
	\item Donner l’équation horaire de la concentration en ions hypochlorite.
	\item Au bout de combien de temps, noté $t_{30}$, aura-t-on obtenu la disparition de $30\%$ des ions hypochlorite ?
	\item Quel serait à $T' = \SI{363}{K}$ le temps $t_{30}'$ nécessaire pour obtenir le même taux d'avancement de $30 \%$ à partir de la même solution initiale ?
\end{enumerate}

\e{Réponses :}
\begin{enumerate}
	\item -
	\item $t_{30} = \SI{23}{min}$.
	\item $t_{30}' = \SI{9}{min}~\SI{20}{s}$.
\end{enumerate}

