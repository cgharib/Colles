\section{Semaine 20 (10/03 - 14/03) }


\e{Notions abordées :}
\begin{itemize}
	\item Filtrage linéaire.
\end{itemize}

\subsection{Questions de cours}

\begin{enumerate}
	\item Grandeur complexe associée à un signal sinusoïdal réel.
	\item Impédances des dipôles usuels. Démonstration.
	\item Exprimer le signal de sortie d'un filtre linéaire en notation réelle en fonction de la fonction de transfert.
\end{enumerate}

\subsection{Exercice 1}

\begin{minipage}[c]{\linewidth/3}
	\begin{circuitikz}
		%Circuit
		\draw (0, 0) 
		to[R, l=$R$] (3, 0)
		to [L, l_=$L$] ++ (0, -2)
		-- (0, -2)
		to [open, v=$v_e$] (0, 0);
		
		\draw (3, 0) --++ (1.5, 0) to [C, l_=$C$, v^<=$v_s$] ++ (0, -2) --(0, -2);
	\end{circuitikz}
\end{minipage}%
\begin{minipage}[c]{\linewidth/2}
	On donne $R = \SI{1.0}{k\Omega}$ et $L = \SI{10}{mH}$.
	\begin{enumerate}
		\item Quel type de filtre ce circuit permet-il de réaliser ?
		\item Déterminer sa fonction de transfert.
		\item Déterminer les pentes des asymptotes en gain BF et HF.
		\item $v_e$ s'écrit comme somme de trois harmoniques de même amplitude, de même phase à l'origine et de fréquences respectives $f_1 = \SI{100}{Hz}$, $f_2 = \SI{1}{kHz}$ et $f_3 = \SI{100}{kHz}$. Écrire $v_e$ puis $v_s$.
		\item $v_e$ est maintenant un triangle de fréquence \SI{60}{Hz}. Quelle est la forme de $v_s$ ?
	\end{enumerate}
\end{minipage}

\subsection{Exercice 2}

\begin{minipage}[c]{\linewidth/2}
	\begin{circuitikz}
		%Circuit
		\draw (0, 0) 
		to[R, l=$R$] (3, 0)
		to[C, l=$C$] ++(2, 0)
		to [L, l=$L$, v<=$v_s$] ++ (0, -2)
		-- (0, -2)
		to [open, v=$v_e$] (0, 0);
	\end{circuitikz}
\end{minipage}%
\begin{minipage}[c]{\linewidth/2}
	\begin{enumerate}
		\item Quel type de filtre ce circuit permet-il de réaliser ?
		\item Déterminer sa fonction de transfert.
		\item Déterminer les pentes des asymptotes en gain BF et HF. Tracer le diagramme de Bode asymptotique.
		\item $v_e$ s'écrit comme somme de trois harmoniques de même amplitude, de même phase à l'origine et de fréquences respectives $f_1 = \SI{100}{Hz}$, $f_2 = \SI{1}{kHz}$ et $f_3 = \SI{100}{kHz}$. Écrire $v_e$ puis $v_s$.
		\item Ce filtre peut-il avoir un comportement dérivateur ? Intégrateur ?
	\end{enumerate}
\end{minipage}

\subsection{Exercice 3}

\begin{minipage}[c]{\linewidth/2}
	\begin{circuitikz}
		%Circuit
		\draw (0, 0) 
		to[R, l=$R$] (2, 0)
		to[C, l=$C$] ++(2, 0)
		to [R, l=$R$] ++ (0, -2)
		-- (0, -2)
		to [open, v=$v_e$] (0, 0);
		\draw (4, 0)
		-- (6, 0)
		to [C, l=$C$, v<=$v_s$] ++ (0, -2)
		-- (4, -2)
		;
	\end{circuitikz}
\end{minipage}%
\begin{minipage}[c]{\linewidth/2}
	On donne $R = \SI{1.0}{k\Omega}$ et $C = \SI{500}{nF}$.
	\begin{enumerate}
		\item Quel type de filtre ce circuit permet-il de réaliser ?
		\item Déterminer sa fonction de transfert.
		\item Déterminer la bande passante. Définir le facteur de qualité.
		\item $v_e$ s'écrit comme somme de trois harmoniques de même amplitude, de même phase à l'origine et de fréquences respectives $f_1 = \SI{100}{Hz}$, $f_2 = \SI{1}{kHz}$ et $f_3 = \SI{100}{kHz}$. Écrire $v_e$ puis $v_s$.
	\end{enumerate}
\end{minipage}