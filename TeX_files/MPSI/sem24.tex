\section{Semaine 23 (07/04 - 11/04) }


\e{Notions abordées :}
\begin{itemize}
	\item Théorème du moment cinétique (cf semaine précédente).
	\item Mouvements à forces centrales.
	\item Diagrammes E-pH.
\end{itemize}

\subsection{Questions de cours}

cf Exercices.

\subsection{Exercice 1 : Expérience de Rutherford}

Dans l'expérience de Rutherford, on bombarde des noyaux d'or ($Z = 79$) de masse $m_{\textrm{Au}} = \SI{3.3e-25}{\kilogram}$ avec un faisceau de particules $\alpha$, c'est-à-dire des noyaux d'hélium de masse $m_\alpha = \SI{6.65e-27}{\kilogram}$. On schématise l'expérience de la façon suivante : le noyau d'or est situé en $O$, la particule $\alpha$, assimilée à un point matériel $M$, arrive à la vitesse $\vec{v_0}$ parallèle à l'axe $(Ox)$ depuis l'infini avec un paramètre d'impact $b$ (distance entre les droites $(Ox)$ et $(M(t=0), \vec{v_0})$). On considère que le noyau d'or reste fixe du fait de sa masse bien supérieure à celle des particules $\alpha$.

\e{Réponses :}

\begin{enumerate}
	\item Quelle force intervient dans cette expérience ? Est-elle répulsive ou attractive ? Donner son expression en fonction de $Z$, $e$, $\epsilon_0$ et de la distance $r$ entre le noyau et la particule. 
	
	\item En déduire que le mouvement est plan.
	
	\item Faire un schéma d'une trajectoire typique en faisant figurer les données de l'énoncé ainsi qu'un système de coordonnées adapté.
	
	\item Montrer que dans ce problème la constante des aires est conservée. L'exprimer en fonction de $b$ et $v_0$.
	
	\item Montrer que l'énergie mécanique $E_m$ de la particule est constante. Introduire ensuite une énergie potentielle effective à exprimer en fonction de $r$, $C$, $m_\alpha$, $Z$, $e$ et $\epsilon_0$. 
	
	\item Tracer le graphe de l'énergie potentielle effective. Justifier qu'il n'y a pas d'état lié.
	
	\item Déterminer l'énergie mécanique initiale en fonction de $m_\alpha$ et $v_0$. En déduire l'ordre de grandeur de la vitesse à laquelle sont envoyées les particules sachant que leur énergie mécanique vaut $E_m = \SI{5.0}{\mega\electronvolt}$. Un traitement classique est-il cohérent ?
	
	\item Dans l'expérience de Rutherford, les expérimentateurs se sont rendus compte que certaines particules repartaient en sens inverse. Considérons une telle particule. Elle subit un choc frontal avec le noyau et donc on suppose $b \approx 0$. En déduire l'énergie mécanique simplifiée. On note $r_\textrm{min}$ la distance minimale d'approche de la particule $\alpha$. En déduire l'expression de $r_{\min}$ en fonction de $Z$, $e$, $m_\alpha$ et $v_0$. En déduire une majoration de la taille du noyau atomique. Commentaire.
\end{enumerate}

\begin{enumerate}
	\item - 
	\item - 
	\item -
	\item - 
	\item -
	\item - 
	\item $v_0 = \SI{1.6e7}{\meter\per\second}$
	\item $r_\textrm{min} = \SI{4.6e-14}{\meter}$
\end{enumerate}

\subsection{Exercice 2 : Force en $1/r^4$}

On considère un point matériel de masse $m$ soumis à une force centrale attractive de norme égale à $K/r^4$ avec $K$ une constante positive.

\begin{enumerate}
	\item Montrer que le mouvement est plan et vérifie la loi des aires.
	\item Définir une énergie potentielle effective. La tracer.
	\item Discuter les différentes trajectoires possibles.
	\item Peut-il y avoir une trajectoire circulaire ? Justifier précisément.
\end{enumerate}

\subsection{Exercice 3 : Modèles planétaires de l'atome d'hydrogène}

Dans le cadre d'un modèle planétaire de l'atome d'hydrogène, l'électron représenté par un point matériel $M$ tourne autour du noyau. Vu le rapport de masse entre l'électron et le proton, on considère le proton fixe dans un référentiel galiléen et l'électron en mouvement. On suppose le mouvement de l'électron non-relativiste.

\begin{enumerate}
	\item Exprimer la force électrostatique subie par l'électron. Justifier que l'interaction gravitationnelle est négligeable.
	
	Dans le cadre du modèle de Bohr, on suppose que la trajectoire de l'électron est circulaire.
	
	\item En déduire que la trajectoire de l'électron est circulaire uniforme. Déterminer la vitesse de l'électron en fonction des données du problème.
	
	\item Exprimer la norme notée $L$ du moment cinétique de $M$ en $O$ en fonction de $r$, $m_e$, $\epsilon_0$ et $e$.
	
	En 1913, Bohr postule la quantification du moment cinétique, c'est-à-dire que $L$ est un multiple entier de $\hbar = \frac{h}{2\pi}$ avec $h = \SI{6.626e-34}{\joule\second}$ la constante de Planck. 
	
	\item Dans ce modèle, montrer que les rayons des orbites de l'électron vérifient $r_n = n^2 a_0$ avec $n \in \mathbb{N}^*$ et $a_0$ le rayon de Bohr que l'on exprimera en fonction de $m$, $\epsilon_0$ et $\hbar$. 
	
	\item Application numérique pour $a_0$. Commentaire. Valider l'hypothèse non-relativiste.
	
	\item Montrer que l'énergie de l'atome d'hydrogène, i.e. l'énergie mécanique de son électron, s'écrit $E_n = -\frac{E_I}{n^2}$ avec $E_I$ la constante énergétique de Rydberg que l'on exprimera en fonction de $m_e$, $\epsilon_0$, $e$ et $\hbar$. Application numérique.
	
	En fait, des mesures de spectres lumineux montre que les énergies de l'atome d'hydrogène ne sont pas exactement prédites par la formule de Bohr. Le modèle peut être complété en considérant des orbites elliptiques. Dans la suite, on admet que la condition de quantification sur le rayon de la trajectoire est toujours valide, à condition de remplacer le rayon par le demi-grand axe de l'ellipse.
	
	\item Écrire l'énergie mécanique du système en faisant apparaître une énergie potentielle effective.
	
	\item Soit $r_a$ (resp. $r_p$) le rayon le plus éloigné (resp. le plus proche) atteint par l'électron lors de sa trajectoire (analogues à l'aphélie et au périhélie dans le mouvement de Kepler). Montrer, grâce à l'expression de l'énergie mécanique, que $r_a$ et $r_p$ sont racines du même polynôme de degré $2$.
	
	\item En déduire l'expression de l'énergie mécanique en fonction du demi-grand axe de l'ellipse $a$. Conclure sur l'expression quantique de l'énergie prédite par ce modèle.
\end{enumerate}

