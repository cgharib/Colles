\section{Semaine 10 (02/12-06/12) }


\e{Notions abordées :}
\begin{itemize}
	\item Propagation d'un signal (cf. semaine précédente).
	\item Phénomènes ondulatoires.
\end{itemize}

\subsection{Questions de cours}

\begin{enumerate}
	\item Relation donnant la demi-largeur angulaire de la tâche centrale de diffraction par une fente de largeur $a$, avec une lumière de longueur d’onde $\lambda$.
	\item Conditions d'interférences constructives et destructives en fonction du déphasage, puis de la différence de marche.
	\item Lien entre l'intensité lumineuse et le signal lumineux. Cas d'un signal sinusoïdal. Pourquoi un capteur lumineux n'est il sensible qu'à l'intensité lumineuse ?
\end{enumerate}

\subsection{Exercice 1 : Interférences acoustiques}

Deux haut-parleurs $S_1$ et $S_2$ alimentés par le même signal $s(t) = s_0 cos (\omega t)$ de fré-
quence $f = \SI{2,0}{kHz}$ sont disposés face à face en $x = 0$ et $x = L$. On place un microphone au point $M$ d’abscisse $x_M$ qui capte l’onde résultant de la superposition des ondes issues des deux émetteurs sachant que les ondes sonores issus de chaque haut parleur se propage à vitesse c et ont une longueur d’onde $\lambda$. En déplaçant le microphone, on repère des zones de l’espace où l’amplitude du signal prend des valeurs maximales.

\begin{enumerate}
	\item Établir les expressions des signaux $s_1(t)$ et $s_2(t)$ correspondant aux ondes acoustiques émises par $S_1$ et $S_2$.
	\item En déduire le déphasage $\phi = \phi_2-\phi_1$ entre les deux ondes au point $M$ d'abscisse $x$ en fonction de $x$, $\lambda$ et $L$.
	\item En déduire les positions $x_c$ des points de l'espace où l'on observe des interférences constructives. Déterminer l'interfrange.
	\item Retrouver ces valeurs en ne raisonnant qu'en terme de trajet parcouru par les ondes.
	\item L'interfrange vaut $\SI{8,6}{cm}$. En déduire la vitesse du son $c$.
\end{enumerate}

\subsection{Exercice 2 : Corde de Melde}

On considère une corde de longueur $L$, fixée en $O$ à gauche et libre d'osciller en $M$ à droite.

\begin{enumerate}
	\item Un opérateur fait osciller l'extrêmité $M$ de la corde verticalement, entre $-z_0$ et $+z_0$ à la pulsation $\omega$. Écrire l'onde progressive qui se propage vers la gauche la vitesse $c$.
	\item On rappelle que la corde est fixée en $O$. Justifier qu'il existe une onde réfléchie et l'écrire.
	\item Calculer l'onde totale. Comment appelle-t-on une telle onde ?
	\item On rappelle que le mouvement est forcé au point $M$. En déduire les pulsations possibles d'oscillation. Commenter.
\end{enumerate}

\subsection{Exercice 3 : La couleur bleue du papillon Morpho}

Une source monochromatique ponctuelle $S$, située dans l'air, de longueur d'onde $\lambda$ éclaire une couche transparente d'indice $n$ et d'épaisseur $e$ en $I$ avec l'angle d'incidence $i$. Un rayon noté $(1)$ est alors réfléchi dans l'air d'indice $n_0$ et un autre rayon $(2)$ est réfracté avec un angle $r$ puis réfléchi en bas de la couche au point $J$, avant d'être réfracté au point $K$ avec un angle $i'$ dans l'air.

Un observateur reçoit les deux rayons lumineux.

On admettra que la notion $(AB)$ désigne la distance $AB$ multipliée par l'indice du milieu dans lesquels sont $A$ et $B$ : $(AB) = n AB$.

\begin{enumerate}
	\item Expliquer pourquoi les rayons $(1)$ et $(2)$ ressortent parallèles. En déduire qu'ils interfèrent sur la rétine.
	\item La différence de marche est donnée par $\delta = (IJ)+(JK)-(IH) + \lambda/2$ où le dernier terme moins évident est lié à la réflexion en $J$. Exprimer $\delta$ en fonction de $e$, $n$, $n_0$, $i$ et $\lambda$.
	\item L'aile du papillon morpho est composé de petites lamelles de chitine d'indice $n=1,7$. On observe une couleur intense dans le bleu pour un angle de vue $i=\SI{30}{\degree}$. Si toutefois on plonge l'aile dans l'acétone d'indice $n_0=1,4$, la couleur est verte. Expliquer ces phénomènes. 
\end{enumerate}
