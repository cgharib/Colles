\section{Semaine 11 (09/12-13/12) }


\e{Notions abordées :}
\begin{itemize}
	\item Phénomènes ondulatoires (cf. semaine précédente).
	\item Structure des entités chimiques.
\end{itemize}

\subsection{Questions de cours}

\begin{enumerate}
	\item Établir de façon détaillée la configuration électronique de l'atome d'azote dans son état fondamental.
	\item Établir de façon détaillée la représentation de Lewis de la molécule de dioxyde de carbone.
	\item Évolution des propriétés dans le tableau périodique (Masse atomique (déf ?), Électronégativité (déf ?), éventuellement rayon atomique.)
\end{enumerate}

\subsection{Exercice : Formules de Lewis}

\begin{enumerate}
	\item \ce{HNO2}, \ce{O3}, \ce{N3-}, \ce{SO2}, \ce{SO3}.
	\item \ce{LiH}, \ce{BeH2}, \ce{BBr3}, \ce{AlN}, \ce{AlCl3}.
\end{enumerate}