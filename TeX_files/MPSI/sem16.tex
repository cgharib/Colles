\section{Semaine 16 (27/01-31/01) }


\e{Notions abordées :}
\begin{itemize}
	\item Travail et énergie (cf. semaine précédente).
	\item Mouvements de particules chargées.
\end{itemize}

\subsection{Questions de cours}

\begin{enumerate}
	\item Relier la variation d'énergie cinétique d'un point matériel dans un champ électrique à la différence de potentiel.
	
	\item Montrer qu'un champ magnétique seul ne permet pas de faire varier l'énergie cinétique d'un point matériel.
	
	\item Déterminer le rayon de la trajectoire d'une charge $q$ dans un champ magnétique.
\end{enumerate}

\subsection{Exercice 1 : Spectromètre de masse}

Des ions positifs de vitesse initiale nulle, de charge $q$ et de masse $m$ issus d'une chambre d'ionisation sont accélérés par une tension $U = \SI{4.0}{kV}$ appliquée entre la sortie de la chambre d'ionisation et une cathode horizontale percée d'un trou $O$.

Au-delà du point $O$, les ions pénètrent dans une zone où règne un champ magnétique $\vec{B}$ perpendiculaire à leur vitesse avec $B = \SI{0.70}{T}$. 

Les ions positifs sont des isotopes $24$ et $26$ des ions \ce{Mg2+}.

\begin{enumerate}
	\item Sur un schéma, faire figurer les différentes zones ainsi que la trajectoire des deux isotopes.
	\item Déterminer la vitesse $v_0$ avec laquelle les ions passent par le trou $O$ en fonction de $U$, $q$ et $m$. 
	\item Quelle est la trajectoire des ions dans le champ magnétique ? On donnera notamment le rayon $R$ de la trajectoire en fonction de $m$, $q$, $B$ et $v_0$ puis en fonction de $m$, $q$, $B$ et $U$.
	\item Calculer la distance $d$ entre le s points d'impact des deux isotopes sur une plaque parallèle au plan du trou $O$.
\end{enumerate}

\subsection{Exercice 2 : Cyclotron}

On considère un cyclotron. Il s'agit d'un dispositif pour accélérer des particules chargées. On s'intéresse à des protons. Il est constituée de trois zones. La zone $1$ occupe l'espace $x < -d/2$. La zone $3$ occupe l'espace $x > d/2$. La zone $2$ est entre les deux.

Dans les zones $1$ et $3$ règne un champ magnétique $\vec{B} = B_0 \vec{e_z}$. Elles servent à faire faire demi-tour aux protons. Dans la zone $2$ règne un champ électrique $\vec{E} = E(t) \vec{e_x}$ avec $E(t)$ qui vaut alternativement $+E_0$ puis $-E_0$ avec une période $T_E$. La zone $2$ sert à accélérer les protons. 

La valeur absolue de la tension dans la zone $2$ vaut $|U| = \SI{100}{kV}$.

On donne $B_0 = \SI{1.47}{T}$.

\begin{enumerate}
	\item Dessiner la trajectoire d'un proton émis avec une vitesse nulle depuis l'interface $1-2$. Faire figurer les champs magnétiques et électriques.
	\item Montrer que la norme de la vitesse est constante dans la zone $3$.
	\item Montrer, en le calculant, que le temps de demi-tour dans la zone $3$ est indépendant de la vitesse du proton.
	\item On néglige le temps de passage dans la zone $2$. En déduire la période $T_E$.
	\item Montrer que le rayon de la n-ème trajectoire dans un champ magnétique est donné par $R_n = R_1 \sqrt{n}$.
	\item Déterminer l'ordonnée $y_n$ du centre de chaque trajectoire circulaire en supposant que le proton est initialement lâché en $y=0$.
	
	Le proton sort du cyclotron lorsqu'il atteint le rayon du cyclotron $\rho = \SI{10.0}{cm}$. 
	
	\item En déduire l'énergie cinétique en $MeV$ d'un proton quand il quitte le cyclotron, le nombre de tours effectués et la durée nécessaire. 
	\item La mécanique classique est-elle toujours valable ?
\end{enumerate}

\e{Réponse :} La durée passée dans le cyclotron se situe entre $t_6 = \SI{1.34e-7}{s}$ et $t_7 = \SI{1.56e-7}{s}$.

\subsection{Exercice 3 : Effet Zeeman}

On considère un électron $M$ de masse $m$ et de charge $-e$ élastiquement lié au noyau $O$ d'un atome par une force de rappel $\vec{F} = -k\overrightarrow{OM}$. 

\begin{enumerate}
	\item Justifier que l'interaction noyau-électron peut effectivement être modélisée par une force de rappel élastique.
	\item Déterminer l'expression de la pulsation de rotation $\omega_0$. 
	\item Des expériences d'absorption d'ondes électromagnétiques montrent que les ondes de longueur d'onde $\lambda = \SI{656.3}{nm}$ sont particulièrement absorbées. En déduire $k$.
	
	On plonge maintenant le système dans un champ magnétique $\vec{B} = B_0 \vec{e_z}$. On supposera $\frac{e B_0}{m \omega_0} \ll 1$. 
	
	\item Projeter les équations du mouvement selon $(Ox)$ et $(Oy)$. 
	\item On cherche des solutions $x(t)$ et $y(t)$ sinusoïdales. Montrer qu'il n'y a que deux pulsations possible. Les exprimer.
	\item En déduire l'écart $\delta \lambda$ entre les deux nouvelles longueur d'onde des ondes possiblement absorbées par ce système.
	\item On suppose que l'on peut mesurer des écarts en longueur d'onde supérieurs à $\delta \lambda_{min} = \SI{0.1}{nm}$. En déduire le champ magnétique minimal que ce dispositif permet de mesurer.
\end{enumerate}