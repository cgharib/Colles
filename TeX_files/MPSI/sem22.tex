\section{Semaine 21 (17/03 - 21/03) }


\e{Notions abordées :}
\begin{itemize}
	\item Réactions de précipitation (cf semaine précédente).
	\item Oxydoréduction.
\end{itemize}

\subsection{Questions de cours}

\begin{enumerate}
	\item Définir un oxydant, un réducteur, une réaction d'oxydoréduction.
	\item Définir l'électrode standard à hydrogène.
	\item Définir le potentiel d'électrode. Équation de Nernst.
\end{enumerate}


\subsection{Exercice 1 : Pile à combustible}

Dans certaines piles à combustible, on utilise le dihydrogène comme combustible et le dioxygène comme comburant. 

\begin{enumerate}
	\item Écrire la réaction de combustion du dihydrogène par le dioxygène.
	
	Cette réaction est en fait l'association de deux demi-équations d'oxydoréduction mettant en jeu deux couples redox.
	
	\item Rappeler les couples redox de l'eau. 
	\item Écrire les deux demi-équations redox.
	
	Les deux demi-réactions ont lieu sur deux électrodes. L'une des électrodes est en contact avec un courant de dihydrogène gazeux à la pression $P^\circ = \SI{1}{bar}$ tandis que l'autre électrode est en contact avec un courant de dihydrogène gazeux à la même pression.
	
	\item Les deux demi-réactions ont lieu sur deux électrodes. Indiquer la réaction cathodique et la réaction anodique.
	\item Donner l'expression du potentiel d'oxydoréduction pour les deux couples. En déduire la tension à vide de la pile.
	\item Exprimer la constante d'équilibre $K^\circ$ en fonction des potentiels standards des couples considérés. Calculer sa valeur et commenter.
\end{enumerate}

\e{Données :} $E^\circ(\ce{H+}/\ce{H2}) = \SI{0.00}{\volt}$ et  $E^\circ(\ce{O2}/\ce{H2O}) = \SI{1.23}{\volt}$.

\e{Réponses :}
\begin{enumerate}
	\item -
	\item -
	\item -
	\item -
	\item $\SI{1.23}{\volt}$
	\item $10^{41}$
\end{enumerate}

\subsection{Exercice 2 : Étude d'une pile de concentration}

Le système réactionnel est une pile électrochimique à $\SI{298}{K}$, utilisant le couple \ce{Fe^3+}/\ce{Fe^2+}. Dans le bécher A, on  initialement $\SI{50}{mL}$ d'une solution à $\SI{0.10}{\mole\per\liter}$ de sulfate ferreux \ce{FeSO4} dissout et $\SI{0.20}{\mole\per\liter}$ de chlorure ferrique \ce{FeCl3} dissout. Dans le bécher B, on a le même volume mais les concentrations inverses. On utilise des électrodes de platine et un pont salin au chlorure de potassium. Les deux béchers sont liés par une très grande résistance électrique.

\begin{enumerate}
	\item Déterminer les concentrations initiales en \ce{Fe^3+} et \ce{Fe^2+} dans les  béchers A et B.
	\item Faire un schéma du dispositif.
	\item Déterminer la différence de potentiel initiale de la pile. Dans quel sens les électrons circulent-ils ? Et le courant électrique. 
	\item Déterminer l'anode et la cathode.
	\item Écrire l'équation bilan traduisant le fonctionnement de la pile. On précisera avec les indices $_A$ et $_B$ le bécher associé à chaque espèce.
	\item Quelle est la valeur de la différence de potentiel lorsque le système n'évolue plus ?
	\item Déterminer les concentrations finales en ions ferreux et ferriques à l'équilibre.
	\item Déterminer la capacité de la pile (charge électrique totale débitée au cours du fonctionnement).
\end{enumerate}

\e{Donnée :} Constante de Faraday $\mathcal{F} = \mathcal{N}_A e = \SI{96500}{\coulomb\per\mole}$.

\e{Réponses :}
\begin{enumerate}
	\item -
	\item -
	\item $\SI{36}{\milli\volt}$
	\item -
	\item -
	\item -
	\item $\SI{0.15}{\mol\per\liter}$
	\item $\SI{241}{\coulomb}$
\end{enumerate}

\subsection{Exercice 3 : Éclairage de l'Opéra de Paris}

En 1874, pour la représentation inaugurale de l'Opéra de Paris, l'éclairage était assuré par un ensemble de piles zinc-chlore : la première électrode est une électrode de zinc solide, plongée dans une solution de sulfate de zinc ; la seconde est une électrode de graphite plongée dans une solution de chlorure de sodium et dans laquelle on fait buller du dichlore gazeux. 

\begin{enumerate}
	\item Déterminer les couples rédox en présence.
	\item Écrire les demi-réactions associées à chaque couple rédox.
	\item Écrire la réaction lorsque la pile débite. Préciser la polarité et le nom de chaque électrode. Faire un schéma.
	\item Donner l'expression des potentiels d'électrode ainsi que de la f.e.m à vide de la pile. Faire l'application numérique pour $[\ce{Zn^2+}]_0 = \SI{0.1}{\mole\per\liter}$, $[\ce{Cl-}]_0 = \SI{0.2}{\mole\per\liter}$ et $p_{\ce{Cl2}} = \SI{1}{bar}$
	\item Quel débit molaire de dichlore faut il entretenir pour une intensité de $\SI{50000}{\ampere}$ (puissance totale de l'ordre de $\SI{100}{\kilo\watt}$). Quel est alors le débit volumique horaire correspondant à $\SI{298}{K}$ à la pression atmosphérique ?
	\item Quelle masse de zinc a été consommée lorsque l'installation a débité $\SI{50000}{\ampere}$ pendant $\SI{4}{\hour}$ ?
	\item Pour toute la batterie de piles, les solutions de sulfate de zinc et de chlorure de sodium occupent $\SI{20}{\meter\cubed}$ chacune. Quelle est la différence de potentiel aux bornes de chaque pile après $\SI{4}{\hour}$ de fonctionnement (une fois le circuit réouvert) ?
\end{enumerate}

\e{Données :}
\begin{itemize}
	\item Masse molaire du zinc $\SI{65.5}{\gram\per\mole}$.
	\item Constante de Faraday $\mathcal{F} = \mathcal{N}_A e = \SI{96500}{\coulomb\per\mole}$.
	\item Potentiels standards $E^\circ(\ce{Cl2(g)}/\ce{Cl-}) = \SI{1.39}{\volt}$ et  $E^\circ(\ce{Zn^2+}/\ce{Zn}) = \SI{-0.76}{\volt}$.
\end{itemize}

\e{Réponses :}
\begin{enumerate}
	\item -
	\item -
	\item -
	\item $\SI{2.22}{\volt}$
	\item $\SI{0.259}{\mol\per\second}$ et $\SI{22.9}{\cubic\meter\per\hour}$
	\item $\SI{3.73}{\mole}$ et $\SI{244}{\kilogram}$
	\item $\SI{2.18}{\volt}$
\end{enumerate}