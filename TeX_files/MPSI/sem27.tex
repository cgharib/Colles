\section{Semaine 27 (12/05 - 16/05) }


\e{Notions abordées :}
\begin{itemize}
	\item Mécanique du solide (cf semaine précédente).
	\item Introduction à la physique quantique.
\end{itemize}

\subsection{Questions de cours}

\begin{enumerate}
	\item Relations de Planck-Einstein.
	\item Longueur d'onde de De Broglie. Interprétation.
	\item Formule de Rydberg pour les longueurs d'ondes d'émission et d'absorption.
\end{enumerate}

\subsection{Exercice 1 : Longueurs d'onde associées à des électrons}

\begin{enumerate}
	\item On considère un électron accéléré par une tension de $\SI{40}{\volt}$.
	\begin{enumerate}
		\item Quelle est son énergie en électron-volts. Est-il classique ou relativiste ?
		\item Quelle est sa longueur d'onde de De Broglie ? Pour quel type d'applications peut-on l'utiliser ?
	\end{enumerate}
	
	\item On considère maintenant un électron accéléré par un accélérateur de particules puissant. Il possède une grande énergie $E = \SI{1}{\giga\electronvolt}$. 
	\begin{enumerate}
		\item Cet électron est il classique ou relativiste ?
		\item Que vaut sa quantité de mouvement (on utilisera la formule relativiste $E^2 = p^2c^2+m^2c^4$) ?
		\item Calculer la longueur d'onde $\lambda$ associée à cet électron. Pour quel type d'applications peut-on l'utiliser ?
	\end{enumerate}  
\end{enumerate}

\subsection{Exercice 2 : Flux de photons dans un faisceau laser}

Un laser Hélium-Néon émet une lumière rouge quasi monochromatique de longueur d'onde dans le vide $\lambda = \SI{633}{\nano\meter}$. La puissance $P$ du faisceau est de $\SI{1.0}{\milli\watt}$ (comme en salle de TP), la section circulaire du faisceau a un diamètre $d = \SI{2.0}{\milli\meter}$.

\begin{enumerate}
	\item Déterminer en $\unit{\joule}$ et en $\unit{\electronvolt}$ l'énergie $E$ d'un photon du faisceau.
	\item Calculer la puissance surfacique $\phi$ qui traverse la section.
	\item Quel est le flux $N_t$ de photons, c'est-à-dire le nombre de photons par seconde dans le faisceau ?
\end{enumerate}

\subsection{Exercice 3 : Télémétrie Terre-Lune}

La mesure de la distance Terre-Lune s'effectue à l'aide d'un laser à impulsions. Le but est de mesurer le temps que met une impulsion pour parcourir un aller-retour entre les deux astres.

Le faisceau laser a une largeur $a = \SI{3.0}{\centi\meter}$, avant d'être envoyé en direction de la Lune. Sur la Lune est disposé un miroir ramené par la mission Apollo XV de 1971, de surface $S = \SI{0.34}{\meter\squared}$.

Le laser à impulsions émet une puissance $P = \SI{1}{\mega\watt}$ pendant un intervalle de temps $\Delta t = \SI{0.3}{\micro\second}$. Sa longueur d'onde est $\lambda = \SI{532}{\nano\meter}$. 

La distance Terre-Lune (surface à surface) vaut $D = \SI{376300}{\kilo\meter}$.

Sur Terre, un détecteur capte les photons réfléchis par le miroir sur la Lune. Il s'agit d'un disque de rayon $r_D = \SI{0.77}{\meter}$.

\begin{enumerate}
	\item Calculer l'énergie $E$ d'un photon du faisceau.
	\item Soit $N_0$ le nombre de photons émis par impulsion laser. Exprimer $N_0$ en fonction de $P$, $\Delta t$ et $E$. Application numérique.
	\item Montrer, en utilisant une relation d'indétermination de Heisenberg, que la largeur du faisceau ne peut qu'augmenter au cours de sa propagation. Comment s'appelle ce phénomène ? Rappeler l'équation qui en donne le demi-angle $\theta$ en fonction de la longueur d'onde $\lambda$ et de la largeur initiale du faisceau $a$.
	\item En déduire la largeur $b$ du faisceau laser lorsqu'il arrive sur la Lune.
	\item Exprimer le nombre $N_1$ de photons qui atteignent le miroir en fonction de $N_0$, $S$ et $b$. Application numérique. 
	\item Exprimer la largeur $c$ du faisceau laser lorsqu'il revient sur Terre et en déduire le nombre de photons $N_2$ captés par le détecteur. Commentaire.
\end{enumerate}
