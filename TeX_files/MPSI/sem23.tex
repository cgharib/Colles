\section{Semaine 23 (31/03 - 04/04) }


\e{Notions abordées :}
\begin{itemize}
	\item Oxydoréduction (cf semaine précédente).
	\item Théorème du moment cinétique.
\end{itemize}

\subsection{Questions de cours}

\begin{enumerate}
	\item Définir l'électrode standard à hydrogène.
	\item Définir le potentiel d'électrode. Équation de Nernst.
	\item Constante d'une réaction rédox.
\end{enumerate}

\subsection{Exercice 1 : Mouvement d'une sphère attachée au bout d'un fil}

Une sphère de petite taille et de masse $m = \SI{0.10}{\kilogram}$ est attachée à l'extrémité d'un fil sans masse de longueur $l_0 = \SI{1.0}{m}$ dont l'autre extrémité est fixée en $O$. Elle se déplace sur un cercle horizontal de rayon $l_0$. Sa vitesse est $v_0 = \SI{1.0}{\meter\per\second}$.

\begin{enumerate}
	\item Déterminer son moment cinétique par rapport à $O$ puis par rapport à $(Oz)$.
	\item On réduit brutalement la longueur du fil à $l_1 = \SI{0.50}{m}$. Que devient la vitesse de la sphère ?
	\item Comparer l'énergie cinétique avant et après la réduction de la longueur du fil. 
	\item Quelle force provoque l'augmentation de l'énergie cinétique de la sphère ? Commenter.
\end{enumerate}

\subsection{Exercice 2 : Esquimau glissant du sommet de son igloo}

Après avoir construit son igloo, un esquimau de masse $m = \SI{70}{\kilogram}$ s'assoit au sommet de ce dernier. On assimile l'igloo à une demi-sphère de centre $O$ et de rayon $R = \SI{1.5}{m}$. Une rafale de vent polaire le déséquilibre légèrement et il glisse du sommet vers le sol. 

\begin{enumerate}
	\item Établir l'équation différentielle du mouvement.
	\item Montrer qu'il n'atteint pas le sol en restant en contact avec l'igloo. On donnera la valeur de l'angle $\alpha$ pour lequel à lieu ce décollage.
\end{enumerate}

\e{Réponses :}
\begin{enumerate}
	\item -
	\item $\alpha = \SI{42}{\degree}$
\end{enumerate}

\subsection{Exercice 3 : Glissement sur un toboggan}

Un enfant, que l'on assimilera à un point matériel $M$ de masse $m = \SI{40}{\kilogram}$ glisse sur un toboggan décrivant une trajectoire circulaire de rayon $r = \SI{2.5}{\meter}$. On négligera l'influence des frottements.

\begin{enumerate}
	\item À l'aide du théorème du moment cinétique, déterminer l'équation du mouvement.
	\item En déduire la vitesse au bout du toboggan. Application numérique.
\end{enumerate}

\e{Réponses :}
\begin{enumerate}
	\item -
	\item $v_f = \SI{6.36}{\meter\per\second}$
\end{enumerate}
