\section{Semaine 21 (17/03 - 21/03) }


\e{Notions abordées :}
\begin{itemize}
	\item Filtrage linéaire (cf semaine précédente).
	\item Réactions de précipitation.
\end{itemize}

\subsection{Questions de cours}

\begin{enumerate}
	\item Définir le produit de solubilité. À quelle condition permet-il effectivement de déterminer la concentration d'espèce dissoute ?
	\item Définir la solubilité. La calculer dans le cas de la dissolution de chlorure d'argent ($K_s = \SI{1.8e-10}{}$).
	\item Définir puis tracer le diagramme d'existence du chlorure d'argent en fonction de $pCl$ pour une concentration $[\ce{Ag+}] = \SI{1.0e-1}{\mole\per\liter}$ ($K_s = \SI{1.8e-10}{}$).
\end{enumerate}


\subsection{Exercice 1 : Précipitation sélective des hydroxydes}

\begin{enumerate}
	\item Lorsque l'on dissout dans l'eau de l'hydroxyde de magnésium(II) jusqu'à saturation, la solution possède un $pH$ égal à $10.5$. Que vaut le produit de solubilité de l'hydroxyde de magnésium(II) ? 
	
	On dispose d'une solution contenant initialement des ions cobalt(II) \ce{Co^2+} à la concentration $c_0 = \SI{1e-2}{\mole\per\liter}$ et des ions magnésium(II) à la même concentration $c_0$. L'objectif est de séparer le cobalt du magnésium. On se donne pour cahier des charges de précipiter plus de $99\%$ du cobalt sans précipiter plus de $1\%$ du magnésium.
	
	\item Calculer la concentration en ions cobalt restant en solution si $99\%$ du cobalt précipite sous forme d'hydroxyde de cobalt. En déduire la concentration en ions hydroxyde, puis le $pH$ de la solution, pour que $99\%$ du cobalt précipite.
	
	\item Calculer la concentration en ions magnésium restant en solution si $1\%$ du magnésium précipite sous forme d'hydroxyde de magnésium. En déduire la concentration en ions hydroxyde, puis le $pH$ de la solution, pour que $99\%$ du cobalt précipite. 
	
	\item Montrer qu'il existe une zone de $pH$ que l'on précisera qui permet de vérifier le cahier des charges.
\end{enumerate}

\e{Donnée :} $pK_s(\ce{Co(OH)2})$ = $\SI{14.8}{}$

\e{Réponses :}
\begin{enumerate}
	\item $pK_{s} = \SI{10.8}{}$
	\item $pH = 8.6$
	\item $pH = 9.6$
	\item -
\end{enumerate}

\subsection{Exercice 2 : Solubilité de l'acide benzoïque}

La réaction de dissolution de l'acide benzoïque dans l'eau s'écrit $$\ce{C6H5COOH_{(s)} = C6H5COOH_{(aq)}}.$$ Son $pKs$ vaut $1.5$ à $\SI{298}{\kelvin}$. La constante d'acidité du couple associé à l'acide benzoïque est déterminée par $pKa = 5$. 

\begin{enumerate}
	\item Calculer la solubilité $s$ de l'acide benzoïque en négligeant son caractère acide.
	\item Calculer la solubilité $s'$ de l'acide benzoïque en tenant compte de ses propriétés acido-basiques. Comparer $s$ et $s'$. Expliquer qualitativement.
	\item Déterminer le $pH$ d'une solution aqueuse saturée d'acide benzoïque.
	
	Le benzoate de sodium est un sel ionique soluble dans l'eau. On dispose d'un volume $V = \SI{1}{L}$ d'une solution de ce sel à la concentration $c_0 = \SI{3.52e-1}{\mole\per\liter}$.
	
	\item Déterminer le $pH$ de précipitation de l'acide benzoïque lors de l'addition d'une solution concentrée d'acide chlorhydrique à la solution précédente. L'acide chlorhydrique étant fortement concentré, on pourra négliger la variation de volume de la solution.
	\item Quelle est la quantité d'acide benzoïque précipité lorsque le $pH$ de la solution vaut $1.0$ ? À quelle masse cela correspond-il ?
\end{enumerate}

\e{Réponses :}
\begin{enumerate}
	\item $\SI{3.2e-2}{\mole\per\liter}$.
	\item $\SI{3.3e-2}{\mole\per\liter}$.
	\item $3.3$
	\item $6$
	\item $\SI{0.320}{\mole}$
\end{enumerate}


\subsection{Exercice 3 : Solubilité de la sidérite}

La solubilité de la sidérite \ce{FeCO3_{(s)}} dans l'eau joue un rôle important dans la composition des lacs ou des eaux souterraines. Les eaux naturelles riches en fer doivent êtres traitées pour la distribution d'eau potable. 

\begin{enumerate}
	\item Le produit de solubilité de la sidérite est $K_s = \SI{1e-11}{}$. Que serait la solubilité de la sidérite dans l'eau en négligeant les propriétés acido-basiques des ions carbonate ?
	
	En fait, la réaction de l'eau sur les ions carbonates ne peut être négligée.
	
	\item On cherche maintenant la solubilité de la sidérite en prenant en compte les propriétés acido-basiques des ions carbonate. En supposant que les ions carbonates ne sont quasiment que sous la forme $\ce{HCO3-}$, déterminer la solubilité de la sidérite. En calculant le $pH$ ainsi que la concentration en $\ce{HCO3-}$, vérifier la cohérence de l'hypothèse ci-dessus.
	
	\item On s'intéresse maintenant à la dissolution du carbonate de fer dans une solution de $pH$ fixé par une solution tampon, ce qui est plus représentatif d'une eau naturelle.
	\begin{enumerate}
		\item Exprimer la solubilité $s$ de la sidérite en fonction des concentrations en les trois formes acido-basiques des ions carbonates.
		\item En déduire $s$ en fonction du $pH$, des constantes d'acidité $K_{a1}$ et $K_{a2}$ et de $K_s$. Commenter l'expression.
		\item Tracer $s(pH)$. On fera aussi figurer trois segments de droites qui représentent $s(pH)$ quand le $pH$ est tel que l'une des formes acido-basiques des carbonates est dominante.
	\end{enumerate}
\end{enumerate}

\e{Données :} $pK_{a1} = 6.4$ et $pK_{a2} = 10.3$ 

\e{Réponses :}
\begin{enumerate}
	\item $\SI{3.2e-6}{\mole\per\liter}$
	\item $\SI{1.3e-5}{\mole\per\liter}$ et $9.1$
	\item -
	\begin{enumerate}
		\item -
		\item $s = \sqrt{K_s\left(1 + \frac{h}{K_{a2}} + \frac{h^2}{K_{a1}K_{a2}} \right))}$
	\end{enumerate}
\end{enumerate}