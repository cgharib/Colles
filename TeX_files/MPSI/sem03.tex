\section{Semaine 03 (30/09-04/10) }


\e{Notions abordées :}
\begin{itemize}
	\item Circuits linéaires du premier ordre (cf semaine précédente).
	\item Équilibre chimique.
\end{itemize}

\subsection{Questions de cours}
\begin{enumerate}
	\item Une mole de méthane réagit avec une mole de dioxygène selon une réaction de combustion. Déterminer la composition finale du système. (Équilibrer + Tableau d'avancement + Avancement final pour une réaction totale).
	\item Exprimer l'activité d'une espèce chimique dans un mélange. Préciser les hypothèses nécessaires.
	\item Exprimer le quotient réactionnel d'une réaction donnée et prévoir le sens d'évolution spontanée d'un système chimique.
\end{enumerate}

\subsection{Exercice 1 : Fluoration du dioxyde d'uranium}

Le dioxyde d'uranium solide réagit avec le fluorure d'hydrogène gazeux pour former du tétrafluorure d'uranium solide et de la vapeur d'eau. 

On maintient la température égale à $\SI{700}{K}$ et la pression totale à $\SI{1}{bar}$. La constante d'équilibre à $\SI{700}{K}$ est égale à $K^\circ = 6.8\times10^4$.

\begin{enumerate}
	\item Écrire la réaction.
	\item On part de $\SI{1.0}{mol}$ de dioxyde d'uranium et de $\SI{1.0}{mol}$ de fluorure d'hydrogène. Quelle sera la composition finale du système ?
	\item Même question en partant de $\SI{0.10}{mol}$ de dioxyde d'uranium et de $\SI{1.0}{mol}$ de fluorure d'hydrogène. Que remarque-t-on dans ce cas ?  
\end{enumerate}

\e{Réponses :}
\begin{enumerate}
	\item -
	\item $\xi = \SI{0.24}{mol}$.
	\item - 
\end{enumerate}

\subsection{Exercice 2 : Constante d'équilibre et quotient de réaction.}

Pour préparer industriellement du dihydrogène, on fait réagir en phase gazeuse du méthane avec de l'eau. La réaction produit également du monoxyde de carbone.

La réaction se déroule sous une pression totale constante $p_{tot} = \SI{10}{bar}$. La constante d'équilibre vaut $K^\circ = 15$. Initialement, le système contient $\SI{10}{mol}$ de méthane, $\SI{30}{mol}$ d'eau, $\SI{5}{mol}$ de monoxyde de carbone et $\SI{15}{mol}$ de dihydrogène. 

\begin{enumerate}
	\item Exprimer la constante d'équilibre en fonction des pressions partielles des constituants.
	\item Exprimer le quotient de réaction $Q$ en fonction de la quantité de matière de chacun des constituants et de la pression totale. Calculer $Q$ dans l'état initial.
	\item Le système est-il à l'équilibre thermodynamique ? Si non, dans quel sens se produira l'évolution ?
	\item Déterminer la composition du système à l'équilibre.
\end{enumerate}

\e{Réponses :}
\begin{enumerate}
	\item -
	\item $Q = 1.56$.
	\item -
	\item $\xi = \SI{3.6}{mol}$.
\end{enumerate}

\subsection{Exercice 3 : Utilisation du quotient de réaction.}

Un récipient de volume $V_0 = \SI{2.00}{L}$ contient initialement $\SI{0.500}{mol}$ de COBr$_2$ qui se décompose à une température de $\SI{346}{K}$ selon la réaction : $$\ce{COBr2_{(g)} = CO_{(g)} + Br2_{(g)}}$$.

\begin{enumerate}
	\item Déterminer la composition du système à l'équilibre sachant que la constante d'équilibre à $\SI{346}{K}$ vaut $K^\circ = 5.46$.
	\item Calculer le pourcentage de COBr$_2$ décomposé à cette température.
	\item L'équilibre précédent étant réalisé, on ajoute $\SI{2.00}{mol}$ de monoxyde de carbone. L'équilibre chimique est-il réalisé ? Si non, décrire l'évolution ultérieure du système.
\end{enumerate}

\e{Réponses :}
\begin{enumerate}
	\item $\xi = \SI{0.285}{mol}$.
	\item 57 \%.
	\item $Q = 43.2$, $\xi' = \SI{0.077}{mol}$.
\end{enumerate}