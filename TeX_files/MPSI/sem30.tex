\section{Semaine 30 (02/06 - 06/06) }


\e{Notions abordées :}
\begin{itemize}
	\item Premier principe de la thermodynamique (cf semaine précédente).
	\item Second principe de la thermodynamique.
\end{itemize}

\subsection{Questions de cours}

\begin{enumerate}
	\item Loi de Laplace, énoncé et démonstration.
	\item Premier principe de la thermodynamique.
	\item Second principe de la thermodynamique.
\end{enumerate}

\subsection{Exercice 1 : Variation d'entropie lors d'un mélange eau-glace}

On mélange une masse $m_e = \SI{5.00}{\kilogram}$ d'eau liquide à la température $T_e = \SI{15.0}{\degreeCelsius}$ et une masse $m_g = \SI{3.00}{\kilogram}$ de glace à la température $T_g = \SI{-45}{\degreeCelsius}$ dans une enceinte calorifugée. On suppose que la pression qui règne dans l'enceinte est la pression atmosphérique au cours de toute l'évolution. On donne la capacité thermique massique de l'eau $c_e = \SI{4.20}{\kilo\joule\per\kilogram\per\kelvin}$, celle de la glace $c_g = \SI{2.15}{\kilo\joule\per\kilogram\per\kelvin}$ et l'enthalpie de fusion de la glace à $T_{fus} = \SI{0.00}{\degreeCelsius}$ soit $L_{fus} = \SI{336}{\kilo\joule\per\kilogram}$.

\begin{enumerate}
	\item Déterminer l'état final du système.
	\item Rappeler le lien entre l'entropie de changement d'état et l'enthalpie de changement d'état.
	\item Calculer la variation d'entropie du système. Commentaire.
	\item Par quel facteur de volume devrait-on détendre $\SI{5.00}{\gram}$ de vapeur d'eau pour obtenir la même variation d'entropie ?
\end{enumerate}

\e{Réponses :}

\begin{enumerate}
	\item $m_g' = \SI{2.93}{\kilogram}$ et $m_e' = \SI{5.07}{\kilogram}$.
	\item -
	\item $\Delta S = \SI{129}{\joule\per\kelvin}$
\end{enumerate}