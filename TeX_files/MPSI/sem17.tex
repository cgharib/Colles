\section{Semaine 17 (03/02-07/02) }


\e{Notions abordées :}
\begin{itemize}
	\item Mouvements de particules chargées (cf. semaine précédente).
	\item Interactions moléculaires.
	\item Équilibres acido-basiques (exercices élémentaires).
\end{itemize}

\subsection{Questions de cours}

\begin{enumerate}
	\item Que sont les interactions de Van der Waals.
	\item Donner l'ordre de grandeur des énergies de liaisons pour la liaison covalente (qq 100 kJ/mol), les interactions de Van der Waals (qq kJ/mol) et la liaison hydrogène (qq 10 kJ/mol). Que signifie la notion d'énergie de liaison ?
	\item Quelles sont les différentes étapes dans le mécanisme de mise en solution d'une espèce chimique (ionisation, dissociation, solvatation). Que sont que la polarité, la proticité et le pouvoir dispersif d'un solvant ? 
\end{enumerate}

\subsection{Exercice 1 : Dissociation d'un acide faible}

L'acide formique de formule \ce{HCO2H} est un monoacide de $pKa$ égal à $3.8$.

\begin{enumerate}
	\item S'agit-il d'un acide fort ou faible ? Justifier.
	\item Dresser le diagramme de prédominance.
	\item Calculer le taux de dissociation $\alpha$ de l'acide d'une solution aqueuse d'acide formique dont la concentration initiale est égale à $c_0 = \SI{1e-1}{\mol\per\liter}$.
	\item Quelle est la valeur du $pH$ lue sur un pH-mètre trempé dans la solution précédente.
\end{enumerate}

\subsection{Exercice 2 : 7.3 p441}

\subsection{Exercice 3 : 7.4 p441}

7.6

