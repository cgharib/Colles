\section{Semaine 13 (06/01-10/01) }


\e{Notions abordées :}
\begin{itemize}
	\item Structure des entités chimiques (cf semaine précédente).
	\item Cinématique du point matériel (cf semaine précédente).
	\item Dynamique du point matériel.
\end{itemize}

\subsection{Questions de cours}

\begin{enumerate}
	\item Définir un référentiel. Définir un référentiel galiléen. Donner des exemples de référentiels galiléens ou non galiléens.
	\item Donner les trois lois de Newton.
	\item Déterminer l'équation différentielle des oscillations d'un pendule simple.
\end{enumerate}

\subsection{Exercice 1 : Descente et chute d'un skieur}

Un skieur descend une piste faisant un angle $\alpha$ avec l’horizontale. On suppose une force de frottement de l’air $\vec{F}$ de norme $\lambda ||\vec{v}||$. On note $\vec{T}$ et $\vec{N}$ les composantes de la réaction de la neige sur les skis et $f$ le coefficient de frottement. On rappelle que s'il y a glissement, on a $||\vec{T}|| = f ||\vec{N}||$.

\begin{enumerate}
	\item Exprimer $T$ et $N$ en fonction des paramètres du problème.
	\item Établir l’équation différentielle en $v(t)$, la vitesse du skieur.
	\item Montrer que le skieur atteint une vitesse limite $v_L$. L'exprimer en fonction de $m$, $\lambda$, $f$, $g$ et $\alpha$. A.N. $\lambda = \SI{1.0}{\newton \second \per \meter}$, $f = 0.9$, $g = \SI{10}{\meter\per\second\squared}$, $m = \SI{80}{kg}$ et $\alpha = \SI{45}{\degree}$.
	\item Calculer la vitesse $v(t)$ et la position $x(t)$ du skieur en fonction de $v_L$, $\lambda$ et $m$.
	\item Calculer l’instant où le skieur atteint une vitesse $v_L/2$.
	\item Il tombe et à partir de ce moment-là, on néglige la résistance de l’air mais le coefficient de frottement est multiplié par $10$. Déterminer la distance parcourue avant de s’arrêter.
\end{enumerate}

\e{Réponses :}
\begin{enumerate}
	\item -
	\item -
	\item $v_L = \SI{57}{\meter\per\second}$
	\item -
	\item $\SI{55}{s}$
	\item $\SI{7.3}{\meter}$
\end{enumerate}

\subsection{Exercice 2 : Un tour en traîneau}

Un traîneau tiré par des chiens se déplaçant sur un sol horizontal est assimilé à un point matériel de masse $M = \SI{5.0e2}{kg}$. La réaction du support est $\vec{R} = \vec{T}+\vec{N}$. $\vec{N}$ est la composante normale à la surface. $\vec{T}$ est la composante tangentielle. On rappelle que 
\begin{itemize}
	\item le traîneau est immobile tant que $||\vec{T}|| < \mu_s ||\vec{N}||$,
	\item s'il y a glissement, $||\vec{T}|| = \mu_d ||\vec{N}||$.
\end{itemize}
avec $\mu_d = \SI{5.0e-2}{}$  et $\mu_s = \SI{8.0e-2}{}$.

On note $\vec{F}$ la force de traction exercée par les chiens sur le traîneau et on admet $||\vec{F}|| = F = F_0 - \beta ||\vec{v}||$ avec $F_0, \beta > 0$.

\begin{enumerate}
	\item Exprimer la valeur minimale de $F_0$ qui permet le démarrage du traîneau. Application numérique. Commentaire.
	\item Le traîneau est en mouvement rectiligne. Déterminer l'équation différentielle sur la vitesse. On définira un temps caractéristique $\tau$.
	\item Exprimer la vitesse limite $v_L$ atteinte par le traîneau en fonction des paramètres du problème.
	\item Déterminer $v(t)$. Tracer son allure. Faire figurer le temps $\tau$.
	\item $v_L$ est atteinte à $5\%$ près au bout d'un temps $t_1 = \SI{5.0}{s}$. Exprimer $\beta$ en fonction de $M$ et $t_1$. Application numérique.
	\item On donne $v_L = \SI{3.0}{\meter\per\second}$. En déduire $F_0$. Commentaire.
	\item Désormais à vitesse constante $v_L$, le traîneau aborde une courbe assimilée à un virage circulaire de rayon $R$ et de centre $O$. Soit $\alpha$ l'angle entre $\vec{F}$ et $\vec{v}$ Déterminer la norme $F$ de la force exercée par les chiens, ainsi que $\tan{\alpha}$ en fonction de $v_L$, $R$, $\mu_d$, $g$ et $M$, afin de maintenir cette trajectoire circulaire.
\end{enumerate}

\e{Réponses :}
\begin{enumerate}
	\item $F_{0, min} = \SI{392}{N}$
	\item -
	\item -
	\item -
	\item $\beta = \SI{300}{kg\per\second}$
	\item $F_0 = \SI{1.1e3}{N}$
	\item $\tan\alpha = \frac{v_L^2}{R \mu_d g}$ et $F = M\sqrt{ \left(\frac{v_L^2}{R}\right)^2 + (\mu_d g)^2}$
\end{enumerate}

\subsection{Exercice 3 : Oscillations amorties d'un plateau}

Une boule de pâte à modeler de masse $m = \SI{250}{g}$ tombe en chute libre d'une hauteur $h_0 = \SI{40}{cm}$ sur un plateau immobile, de masse négligeable et supporté par un ressort vertical. On considère qu'au moment du contact il n'y a pas de perte d'énergie cinétique, c'est-à-dire que la boule à la même vitesse juste avant le contact, et juste après lorsqu'elle est solidaire du plateau et se met à osciller avec lui.

L'origine des altitudes est prise à la position initiale du plateau.

\begin{enumerate}
	\item Sachant que le ressort a pour raideur $k = \SI{500}{N\per\meter}$, déterminer la hauteur $h_1$ dont s'affaisse le plateau. Quelle est la hauteur maximale $h_2$ atteinte par la boule lors des oscillations ? Commentaire.
	\item En réalité, les oscillations sont amorties et le système finit par s'immobiliser. Calculer la hauteur à l'équilibre $h_e$.
	\item On suppose que le frottement peut être modélisée par une force $\vec{F}$ de norme $||\vec{F}|| = \lambda ||\vec{v}||$. Le plateau s'immobilise (à $5\%$ près et à altitude $h_e$) au bout de $t = \SI{10}{s}$. Déterminer l'équation horaire de la trajectoire ainsi que la valeur de $\lambda$.
\end{enumerate}

\subsection{Exercice 4 : Pendule contrarié}

Une masse ponctuelle $m$ est accrochée à l'aide d'un fil sans masse de longueur $l$ au point fixe $O$. On la lâche avec une vitesse nulle et avec un angle $\theta_0$. En dessous de $O$, à une distance $h<l$, est fixé un clou, de section négligeable. On suppose que $m$ a la même vitesse juste avant et juste après le contact du fil en $A$.

À quelle condition sur $\theta_0$, la masse fait elle un tour entier autour de $A$, fil tendu ? 

