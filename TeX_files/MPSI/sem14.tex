\section{Semaine 14 (13/01-17/01) }


\e{Notions abordées :}
\begin{itemize}
	\item Dynamique du point matériel (cf. semaine précédente).
	\item Travail et énergie.
\end{itemize}

\subsection{Questions de cours}

\begin{enumerate}
	\item Démontrer le théorème de la puissance cinétique.
	\item Définir une force conservative. Calculer l'énergie potentielle d'un ressort.
	\item Comment l'énergie potentielle détermine-t-elle les positions d'équilibre et leur stabilité ?
\end{enumerate}

\subsection{Exercice 1 : Le sport fait-il maigrir ?}

On considère un cycliste assimilé à un point matériel de masse $m=\SI{90}{kg}$ montant une côte modélisée par un segment $OA$ incliné d'un angle $\alpha$. Le dénivelé entre $O$ et $A$ est noté $h = \SI{50}{m}$ et la distance parcourue par le vélo dans la côte $L=\SI{1000}{m}$.

On note $g=\SI{9.8}{\meter\per\second\squared}$ l'accélération de pesanteur.

Les frottements de l'air sont modélisés par une force $\overrightarrow{f_air} = -\lambda ||\vec{v}|| \vec{v}$ avec $\lambda = \SI{0.21}{SI}$.

Le cycliste subit également une force de frottement solide s'opposant à son mouvement. On note sa composante tangentielle $\vec{T}$, sa composante normale $\vec{N}$. On rappelle que $||\vec{T}|| = f ||\vec{N}||$ avec $f = \SI{8.0e-3}{}$.

On note $\vec{F}$ la force motrice que le cycliste déploie en pédalant pour faire monter le vélo dans la direction de la pente.

Le cycliste monte la côte à vitesse constante.

\begin{enumerate}
	\item Quelle est l'unité de $\lambda$ ?
	\item Déterminer $\alpha$.
	\item Faire un schéma en faisant figurer les différentes forces.
	\item Déterminer la valeur de la force motrice en fonction de la vitesse. Application numérique pour une vitesse $v_1 = \SI{15}{\kilo\meter\per\hour}$.
	\item En déduire la puissance $P_{\textrm{montee}}$ développée par la force motrice dans cette montée. Sachant que le rendement mécanique des muscles du corps humain n'est que de l'ordre de $\eta = 23\%$, déterminer la puissance effectivement fournie par le cycliste. 
	
	Dans la suite, on souhaite retrouver ce résultat par une méthode purement énergétique.
	\item Exprimer l'énergie potentielle de pesanteur $E_p$. En déduire la variation d'énergie mécanique $\Delta E_m$ entre le début et la fin de la montée effectuée à vitesse constante.
	\item Calculer le travail de la force de frottement solide $\vec{T}$. Que vaut le travail de la composante normale ? Exprimer le travail des frottements de l'air.
	\item En déduire le travail développé par la force motrice. Application numérique. Quelle est l'énergie dépensée par le cycliste, en prenant compte le rendement musculaire $\eta$ ? En déduire à nouveau la puissance fournie par le cycliste.
	\item Sachant qu'une calorie vaut $\SI{4.18}{J}$ et que la consommation de $\SI{100}{g}$ de sucre fournit une énergie de $\SI{400}{kcal}$, estimer la masse de sucre à ingérer pour fournir cet effort durant le temps $t_1$. Sachant que $\SI{100}{g}$ de lipides fournissent $\SI{900}{kcal}$ au corps humain, quelle est la masse maximale de graisse brûlée lors de cette montée ?
\end{enumerate}

\e{Réponse :} $\SI{14}{g}$ de sucre ou $\SI{6.3}{g}$ de graisse.

\subsection{Exercice 2 : Molécule diatomique}

On considère une molécule diatomique formée de deux atomes $M_1$ et $M_2$ ponctuels, partiellement ionisés et de charges respectives $q_1 = +\delta e$ et $q_2 = -\delta e$.

On suppose $M_1$ immobile dans un référentiel galiléen, à l'origine d'un repère cartésien, et l'étude concerne le mouvement de $M_2$ selon l'axe $(Ox)$.

L'énergie potentielle de $M_2$ est bien représentée par $$E_p(x) = -\frac{\delta^2 e^2}{4\pi\epsilon_0 x} + \frac{A}{x^9},$$ avec $A$ constante positive.

\begin{enumerate}
	\item Indiquer si chacun des termes de l'énergie potentielle correspond à une force attractive ou répulsive. Donner une signification physique de ces termes.
	\item Tracer l'allure de $E_p(x)$.
	\item Donner la valeur $x_e$ de la position à l'équilibre. L'équilibre est-il stable ?
	\item En s'intéressant aux petites oscillations autour de la position d'équilibre, déterminer la fréquence de vibration de la molécule en fonction de $\delta e$, $\epsilon_0$, $x_e$ et $m$.
\end{enumerate}

\subsection{Exercice 3 : Bifurcation mécanique}

On s'intéresse à une bifurcation, c'est-à-dire une modification du nombre de positions d'équilibre et de leur stabilité, en fonction de la variation d'un paramètre expérimental.

Le système considéré est constitué d'un point matériel $M$ de masse $m$. Il est astreint à se déplacer selon un axe horizontal $(Ox)$ est est fixé à l'extrémité d'un ressort $(k, l_0)$. L'autre extremité $R$ du ressort est fixée à une altitude $l$ par rapport à l'origine $O$.

On posera $\omega_0 = \sqrt{\frac{k}{m}}$.

\begin{enumerate}
	\item Faire un schéma précis du montage.
	\item Qualitativement, déterminer le nombre de positions d'équilibre dans les cas $l > l_0$ et $l < l_0$.
	\item On se place à $l$ quelconque. Déterminer l'expression de l'énergie potentielle du système à partir du calcul du travail élémentaire des forces.
	\item Retrouver ce résultat en explicitant l'énergie potentielle élastique associée au ressort.
	\item Déterminer les expressions des positions d'équilibre en distinguant les cas $l>l_0$ et $l<l_0$.
	\item Pour chacune des positions d'équilibre trouvée, étudier sa stabilité.
	\item Tracer sur un même graphe les positions d'équilibre en fonction de $l$ en précisant leur stabilité. Justifier le nom de bifurcation fourche donné à cette situation.
	\item On dit également qu'il s'agit d'une bifurcation à brisure de symétrie. Justifier cette expression. 
\end{enumerate}