\section{Semaine 29 (26/05 - 30/05) }


\e{Notions abordées :}
\begin{itemize}
	\item Corps pur diphasé (cf semaine précédente).
	\item Premier principe de la thermodynamique.
\end{itemize}

\subsection{Questions de cours}

\begin{enumerate}
	\item Premier principe de la thermodynamique.
	\item Détente de Joule-Gay-Lussac.
	\item Loi de Laplace, démonstration.
\end{enumerate}

\subsection{Exercice 1 : Compressions adiabatiques d'un gaz parfait}

Un piston de masse négligeable ferme un cylindre aux parois calorifugés contenant un gaz parfait sous une pression $P_0$ à une température $T_0$ et occupant un volume $V_0$. L'atmosphère est à la pression $P_0$ et à la température $T_0$. On pose une masse $M$ sur le piston de sorte que le gaz subit une compression brutale conduisant à un état d'équilibre à la pression $P_1 = 2P_0$, à la température $T_1$ et au volume $V_1$. On note $n$ la quantité de matière du gaz et $R$ la constante des gaz parfaits. On supposera le gaz parfait et diatomique.

\begin{enumerate}
	\item Comment qualifier la transformation subie par le gaz ?
	\item Calculer le travail des forces de pression $W$ en fonction de $P_0$, $V_1$ et $V_0$.
	\item En déduire l'expression du travail des forces de pression en fonction de $n$, $R$, $T_1$ et $T_0$.
	\item En appliquant le premier principe, déterminer une autre expression du travail $W$.
	\item En déduire $T_1$ en fonction de $T_0$.
	\item Si on enlève la masse, montrer que la température $T_2$ du nouvel état d'équilibre n'est pas $T_0$.
	\item Que vaut le coefficient adiabatique $\gamma$ pour ce gaz ? Rappeler les conditions d'application de la loi de Laplace.
	\item Repartant du même état initial à $P_0$, $T_0$ et $V_0$, déterminer la température $T_3$ que l'on aurait atteinte si on avait comprimé lentement le gaz en augmentant progressivement la pression jusqu'à $P_3 = 2P_0$. Commentaire.
\end{enumerate}

\e{Réponses :}
\begin{enumerate}
	\item -
	\item -
	\item -
	\item -
	\item $T_1 = \frac{9}{7}T_0$
	\item $T_2 = \frac{54}{49}T_0$
	\item -
	\item $T_3 = 1.2 T_0$
\end{enumerate}


\subsection{Exercice 2 : Détermination de la capacité thermique d'un métal}

\begin{enumerate}
	\item On mélange $\SI{95}{\gram}$ d'eau à la température de $\SI{20}{\degreeCelsius}$ et $\SI{170}{\gram}$ d'eau à la température de $\SI{47}{\degreeCelsius}$. En ne tenant pas compte de la capacité thermique des instruments, calculer la température du mélange lorsque l'équilibre thermique est établi.
	\item On mesure une température de $\SI{35}{\degreeCelsius}$. Que peut-on en conclure ? Calculer la valeur en eau des instruments, c'est-à-dire la masse d'eau $\mu$ qui aurait la même capacité thermique que les instruments, en supposant qu'on a d'abord versé la masse d'eau la plus froide et qu'il y a eu équilibre avec les instruments.
	\item On plonge maintenant dans le calorimètre contenant $\SI{100}{\gram}$ d'eau à la température de $\SI{20}{\degreeCelsius}$ une barre de métal à la température $\SI{60}{\degreeCelsius}$. La barre a une masse de $\SI{200}{\gram}$. À l'équilibre, on mesure une température de $\SI{30}{\degreeCelsius}$. Déterminer la capacité thermique massique du métal. On donne celle de l'eau $c_e = \SI{4.18}{\kilo\joule\per\kilogram\per\kelvin}$. Commentaire.
\end{enumerate}

\e{Réponses :}
\begin{enumerate}
	\item $T_f = \SI{37}{\degreeCelsius}$.
	\item $\mu = \SI{41}{\gram}$.
	\item $c_m = \SI{982}{\joule\per\kilogram\per\kelvin}$
\end{enumerate}

\subsection{Exercice 3 : Vaporisation totale d'un mélange liquide-vapeur}

On s'intéresse à un mélange liquide-vapeur de masse $m = \SI{30}{\gram}$ contenu dans une enceinte de volume $V_A$ à la température $T_1 = \SI{100}{\degreeCelsius}$. On donne $l_{vap}{T_1} = \SI{2.3}{\mega\joule\per\kilogram}$ l'enthalpie massique de vaporisation et $P_{sat}(T_1) = P_1 = \SI{1.0}{bar}$ la pression de vapeur saturante à cette température.

\begin{enumerate}
	\item Tracer le diagramme de Clapeyron donnant la pression en fonction du volume massique. Nommer les différentes courbes qui y apparaissent. Tracer l'isotherme $T_1$. Placer $P_1$. Placer le point $A$ correspondant à l'état initial du système en sachant qu'il est composé de $\SI{20}{\percent}$ de liquide en masse.
	\item On donne $v_l = \SI{1.0e-3}{\meter\cubed\per\kilogram}$ et $v_v = \SI{1.7}{\meter\cubed\per\kilogram}$. Après avoir commenté ces deux valeurs, déduire le volume massique $v_A$ du mélange puis le volume $V_A$ correspondant.
	\item On effectue une vaporisation isobare totale $AB$ jusqu'à obtenir de la vapeur saturante en augmentant lentement le volume disponible. Quelle est la température de la vapeur saturante ? Représenter la transformation $AB$ sur le graphe. Déterminer le volume $V_B$ final.
	\item En déduire les expressions et les valeurs de $W_{AB}$ puis $Q_{AB}$.
\end{enumerate}

\e{Réponses :}
\begin{enumerate}
	\item -
	\item $v_A = \SI{1.36}{\meter\cubed\per\kilogram}$ et $V_A = \SI{0.041}{\meter\cubed}$.
	\item $V_B = \SI{0.051}{\meter\cubed}$
	\item $W_{AB} = \SI{-1.0}{\kilo\joule}$ et $Q_{AB} = \SI{14}{\kilo\joule}$ 
\end{enumerate}
